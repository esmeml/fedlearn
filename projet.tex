\documentclass[11pt]{article}

    \usepackage[breakable]{tcolorbox}
    \usepackage{parskip} % Stop auto-indenting (to mimic markdown behaviour)
    
    \usepackage{iftex}
    \ifPDFTeX
    	\usepackage[T1]{fontenc}
    	\usepackage{mathpazo}
    \else
    	\usepackage{fontspec}
    \fi

    % Basic figure setup, for now with no caption control since it's done
    % automatically by Pandoc (which extracts ![](path) syntax from Markdown).
    \usepackage{graphicx}
    % Maintain compatibility with old templates. Remove in nbconvert 6.0
    \let\Oldincludegraphics\includegraphics
    % Ensure that by default, figures have no caption (until we provide a
    % proper Figure object with a Caption API and a way to capture that
    % in the conversion process - todo).
    \usepackage{caption}
    \DeclareCaptionFormat{nocaption}{}
    \captionsetup{format=nocaption,aboveskip=0pt,belowskip=0pt}

    \usepackage[Export]{adjustbox} % Used to constrain images to a maximum size
    \adjustboxset{max size={0.9\linewidth}{0.9\paperheight}}
    \usepackage{float}
    \floatplacement{figure}{H} % forces figures to be placed at the correct location
    \usepackage{xcolor} % Allow colors to be defined
    \usepackage{enumerate} % Needed for markdown enumerations to work
    \usepackage{geometry} % Used to adjust the document margins
    \usepackage{amsmath} % Equations
    \usepackage{amssymb} % Equations
    \usepackage{textcomp} % defines textquotesingle
    % Hack from http://tex.stackexchange.com/a/47451/13684:
    \AtBeginDocument{%
        \def\PYZsq{\textquotesingle}% Upright quotes in Pygmentized code
    }
    \usepackage{upquote} % Upright quotes for verbatim code
    \usepackage{eurosym} % defines \euro
    \usepackage[mathletters]{ucs} % Extended unicode (utf-8) support
    \usepackage{fancyvrb} % verbatim replacement that allows latex
    \usepackage{grffile} % extends the file name processing of package graphics 
                         % to support a larger range
    \makeatletter % fix for grffile with XeLaTeX
    \def\Gread@@xetex#1{%
      \IfFileExists{"\Gin@base".bb}%
      {\Gread@eps{\Gin@base.bb}}%
      {\Gread@@xetex@aux#1}%
    }
    \makeatother

    % The hyperref package gives us a pdf with properly built
    % internal navigation ('pdf bookmarks' for the table of contents,
    % internal cross-reference links, web links for URLs, etc.)
    \usepackage{hyperref}
    % The default LaTeX title has an obnoxious amount of whitespace. By default,
    % titling removes some of it. It also provides customization options.
    \usepackage{titling}
    \usepackage{longtable} % longtable support required by pandoc >1.10
    \usepackage{booktabs}  % table support for pandoc > 1.12.2
    \usepackage[inline]{enumitem} % IRkernel/repr support (it uses the enumerate* environment)
    \usepackage[normalem]{ulem} % ulem is needed to support strikethroughs (\sout)
                                % normalem makes italics be italics, not underlines
    \usepackage{mathrsfs}
    

    
    % Colors for the hyperref package
    \definecolor{urlcolor}{rgb}{0,.145,.698}
    \definecolor{linkcolor}{rgb}{.71,0.21,0.01}
    \definecolor{citecolor}{rgb}{.12,.54,.11}

    % ANSI colors
    \definecolor{ansi-black}{HTML}{3E424D}
    \definecolor{ansi-black-intense}{HTML}{282C36}
    \definecolor{ansi-red}{HTML}{E75C58}
    \definecolor{ansi-red-intense}{HTML}{B22B31}
    \definecolor{ansi-green}{HTML}{00A250}
    \definecolor{ansi-green-intense}{HTML}{007427}
    \definecolor{ansi-yellow}{HTML}{DDB62B}
    \definecolor{ansi-yellow-intense}{HTML}{B27D12}
    \definecolor{ansi-blue}{HTML}{208FFB}
    \definecolor{ansi-blue-intense}{HTML}{0065CA}
    \definecolor{ansi-magenta}{HTML}{D160C4}
    \definecolor{ansi-magenta-intense}{HTML}{A03196}
    \definecolor{ansi-cyan}{HTML}{60C6C8}
    \definecolor{ansi-cyan-intense}{HTML}{258F8F}
    \definecolor{ansi-white}{HTML}{C5C1B4}
    \definecolor{ansi-white-intense}{HTML}{A1A6B2}
    \definecolor{ansi-default-inverse-fg}{HTML}{FFFFFF}
    \definecolor{ansi-default-inverse-bg}{HTML}{000000}

    % commands and environments needed by pandoc snippets
    % extracted from the output of `pandoc -s`
    \providecommand{\tightlist}{%
      \setlength{\itemsep}{0pt}\setlength{\parskip}{0pt}}
    \DefineVerbatimEnvironment{Highlighting}{Verbatim}{commandchars=\\\{\}}
    % Add ',fontsize=\small' for more characters per line
    \newenvironment{Shaded}{}{}
    \newcommand{\KeywordTok}[1]{\textcolor[rgb]{0.00,0.44,0.13}{\textbf{{#1}}}}
    \newcommand{\DataTypeTok}[1]{\textcolor[rgb]{0.56,0.13,0.00}{{#1}}}
    \newcommand{\DecValTok}[1]{\textcolor[rgb]{0.25,0.63,0.44}{{#1}}}
    \newcommand{\BaseNTok}[1]{\textcolor[rgb]{0.25,0.63,0.44}{{#1}}}
    \newcommand{\FloatTok}[1]{\textcolor[rgb]{0.25,0.63,0.44}{{#1}}}
    \newcommand{\CharTok}[1]{\textcolor[rgb]{0.25,0.44,0.63}{{#1}}}
    \newcommand{\StringTok}[1]{\textcolor[rgb]{0.25,0.44,0.63}{{#1}}}
    \newcommand{\CommentTok}[1]{\textcolor[rgb]{0.38,0.63,0.69}{\textit{{#1}}}}
    \newcommand{\OtherTok}[1]{\textcolor[rgb]{0.00,0.44,0.13}{{#1}}}
    \newcommand{\AlertTok}[1]{\textcolor[rgb]{1.00,0.00,0.00}{\textbf{{#1}}}}
    \newcommand{\FunctionTok}[1]{\textcolor[rgb]{0.02,0.16,0.49}{{#1}}}
    \newcommand{\RegionMarkerTok}[1]{{#1}}
    \newcommand{\ErrorTok}[1]{\textcolor[rgb]{1.00,0.00,0.00}{\textbf{{#1}}}}
    \newcommand{\NormalTok}[1]{{#1}}
    
    % Additional commands for more recent versions of Pandoc
    \newcommand{\ConstantTok}[1]{\textcolor[rgb]{0.53,0.00,0.00}{{#1}}}
    \newcommand{\SpecialCharTok}[1]{\textcolor[rgb]{0.25,0.44,0.63}{{#1}}}
    \newcommand{\VerbatimStringTok}[1]{\textcolor[rgb]{0.25,0.44,0.63}{{#1}}}
    \newcommand{\SpecialStringTok}[1]{\textcolor[rgb]{0.73,0.40,0.53}{{#1}}}
    \newcommand{\ImportTok}[1]{{#1}}
    \newcommand{\DocumentationTok}[1]{\textcolor[rgb]{0.73,0.13,0.13}{\textit{{#1}}}}
    \newcommand{\AnnotationTok}[1]{\textcolor[rgb]{0.38,0.63,0.69}{\textbf{\textit{{#1}}}}}
    \newcommand{\CommentVarTok}[1]{\textcolor[rgb]{0.38,0.63,0.69}{\textbf{\textit{{#1}}}}}
    \newcommand{\VariableTok}[1]{\textcolor[rgb]{0.10,0.09,0.49}{{#1}}}
    \newcommand{\ControlFlowTok}[1]{\textcolor[rgb]{0.00,0.44,0.13}{\textbf{{#1}}}}
    \newcommand{\OperatorTok}[1]{\textcolor[rgb]{0.40,0.40,0.40}{{#1}}}
    \newcommand{\BuiltInTok}[1]{{#1}}
    \newcommand{\ExtensionTok}[1]{{#1}}
    \newcommand{\PreprocessorTok}[1]{\textcolor[rgb]{0.74,0.48,0.00}{{#1}}}
    \newcommand{\AttributeTok}[1]{\textcolor[rgb]{0.49,0.56,0.16}{{#1}}}
    \newcommand{\InformationTok}[1]{\textcolor[rgb]{0.38,0.63,0.69}{\textbf{\textit{{#1}}}}}
    \newcommand{\WarningTok}[1]{\textcolor[rgb]{0.38,0.63,0.69}{\textbf{\textit{{#1}}}}}
    
    
    % Define a nice break command that doesn't care if a line doesn't already
    % exist.
    \def\br{\hspace*{\fill} \\* }
    % Math Jax compatibility definitions
    \def\gt{>}
    \def\lt{<}
    \let\Oldtex\TeX
    \let\Oldlatex\LaTeX
    \renewcommand{\TeX}{\textrm{\Oldtex}}
    \renewcommand{\LaTeX}{\textrm{\Oldlatex}}
    % Document parameters
    % Document title
    \title{projet}
    
    
    
    
    
% Pygments definitions
\makeatletter
\def\PY@reset{\let\PY@it=\relax \let\PY@bf=\relax%
    \let\PY@ul=\relax \let\PY@tc=\relax%
    \let\PY@bc=\relax \let\PY@ff=\relax}
\def\PY@tok#1{\csname PY@tok@#1\endcsname}
\def\PY@toks#1+{\ifx\relax#1\empty\else%
    \PY@tok{#1}\expandafter\PY@toks\fi}
\def\PY@do#1{\PY@bc{\PY@tc{\PY@ul{%
    \PY@it{\PY@bf{\PY@ff{#1}}}}}}}
\def\PY#1#2{\PY@reset\PY@toks#1+\relax+\PY@do{#2}}

\expandafter\def\csname PY@tok@w\endcsname{\def\PY@tc##1{\textcolor[rgb]{0.73,0.73,0.73}{##1}}}
\expandafter\def\csname PY@tok@c\endcsname{\let\PY@it=\textit\def\PY@tc##1{\textcolor[rgb]{0.25,0.50,0.50}{##1}}}
\expandafter\def\csname PY@tok@cp\endcsname{\def\PY@tc##1{\textcolor[rgb]{0.74,0.48,0.00}{##1}}}
\expandafter\def\csname PY@tok@k\endcsname{\let\PY@bf=\textbf\def\PY@tc##1{\textcolor[rgb]{0.00,0.50,0.00}{##1}}}
\expandafter\def\csname PY@tok@kp\endcsname{\def\PY@tc##1{\textcolor[rgb]{0.00,0.50,0.00}{##1}}}
\expandafter\def\csname PY@tok@kt\endcsname{\def\PY@tc##1{\textcolor[rgb]{0.69,0.00,0.25}{##1}}}
\expandafter\def\csname PY@tok@o\endcsname{\def\PY@tc##1{\textcolor[rgb]{0.40,0.40,0.40}{##1}}}
\expandafter\def\csname PY@tok@ow\endcsname{\let\PY@bf=\textbf\def\PY@tc##1{\textcolor[rgb]{0.67,0.13,1.00}{##1}}}
\expandafter\def\csname PY@tok@nb\endcsname{\def\PY@tc##1{\textcolor[rgb]{0.00,0.50,0.00}{##1}}}
\expandafter\def\csname PY@tok@nf\endcsname{\def\PY@tc##1{\textcolor[rgb]{0.00,0.00,1.00}{##1}}}
\expandafter\def\csname PY@tok@nc\endcsname{\let\PY@bf=\textbf\def\PY@tc##1{\textcolor[rgb]{0.00,0.00,1.00}{##1}}}
\expandafter\def\csname PY@tok@nn\endcsname{\let\PY@bf=\textbf\def\PY@tc##1{\textcolor[rgb]{0.00,0.00,1.00}{##1}}}
\expandafter\def\csname PY@tok@ne\endcsname{\let\PY@bf=\textbf\def\PY@tc##1{\textcolor[rgb]{0.82,0.25,0.23}{##1}}}
\expandafter\def\csname PY@tok@nv\endcsname{\def\PY@tc##1{\textcolor[rgb]{0.10,0.09,0.49}{##1}}}
\expandafter\def\csname PY@tok@no\endcsname{\def\PY@tc##1{\textcolor[rgb]{0.53,0.00,0.00}{##1}}}
\expandafter\def\csname PY@tok@nl\endcsname{\def\PY@tc##1{\textcolor[rgb]{0.63,0.63,0.00}{##1}}}
\expandafter\def\csname PY@tok@ni\endcsname{\let\PY@bf=\textbf\def\PY@tc##1{\textcolor[rgb]{0.60,0.60,0.60}{##1}}}
\expandafter\def\csname PY@tok@na\endcsname{\def\PY@tc##1{\textcolor[rgb]{0.49,0.56,0.16}{##1}}}
\expandafter\def\csname PY@tok@nt\endcsname{\let\PY@bf=\textbf\def\PY@tc##1{\textcolor[rgb]{0.00,0.50,0.00}{##1}}}
\expandafter\def\csname PY@tok@nd\endcsname{\def\PY@tc##1{\textcolor[rgb]{0.67,0.13,1.00}{##1}}}
\expandafter\def\csname PY@tok@s\endcsname{\def\PY@tc##1{\textcolor[rgb]{0.73,0.13,0.13}{##1}}}
\expandafter\def\csname PY@tok@sd\endcsname{\let\PY@it=\textit\def\PY@tc##1{\textcolor[rgb]{0.73,0.13,0.13}{##1}}}
\expandafter\def\csname PY@tok@si\endcsname{\let\PY@bf=\textbf\def\PY@tc##1{\textcolor[rgb]{0.73,0.40,0.53}{##1}}}
\expandafter\def\csname PY@tok@se\endcsname{\let\PY@bf=\textbf\def\PY@tc##1{\textcolor[rgb]{0.73,0.40,0.13}{##1}}}
\expandafter\def\csname PY@tok@sr\endcsname{\def\PY@tc##1{\textcolor[rgb]{0.73,0.40,0.53}{##1}}}
\expandafter\def\csname PY@tok@ss\endcsname{\def\PY@tc##1{\textcolor[rgb]{0.10,0.09,0.49}{##1}}}
\expandafter\def\csname PY@tok@sx\endcsname{\def\PY@tc##1{\textcolor[rgb]{0.00,0.50,0.00}{##1}}}
\expandafter\def\csname PY@tok@m\endcsname{\def\PY@tc##1{\textcolor[rgb]{0.40,0.40,0.40}{##1}}}
\expandafter\def\csname PY@tok@gh\endcsname{\let\PY@bf=\textbf\def\PY@tc##1{\textcolor[rgb]{0.00,0.00,0.50}{##1}}}
\expandafter\def\csname PY@tok@gu\endcsname{\let\PY@bf=\textbf\def\PY@tc##1{\textcolor[rgb]{0.50,0.00,0.50}{##1}}}
\expandafter\def\csname PY@tok@gd\endcsname{\def\PY@tc##1{\textcolor[rgb]{0.63,0.00,0.00}{##1}}}
\expandafter\def\csname PY@tok@gi\endcsname{\def\PY@tc##1{\textcolor[rgb]{0.00,0.63,0.00}{##1}}}
\expandafter\def\csname PY@tok@gr\endcsname{\def\PY@tc##1{\textcolor[rgb]{1.00,0.00,0.00}{##1}}}
\expandafter\def\csname PY@tok@ge\endcsname{\let\PY@it=\textit}
\expandafter\def\csname PY@tok@gs\endcsname{\let\PY@bf=\textbf}
\expandafter\def\csname PY@tok@gp\endcsname{\let\PY@bf=\textbf\def\PY@tc##1{\textcolor[rgb]{0.00,0.00,0.50}{##1}}}
\expandafter\def\csname PY@tok@go\endcsname{\def\PY@tc##1{\textcolor[rgb]{0.53,0.53,0.53}{##1}}}
\expandafter\def\csname PY@tok@gt\endcsname{\def\PY@tc##1{\textcolor[rgb]{0.00,0.27,0.87}{##1}}}
\expandafter\def\csname PY@tok@err\endcsname{\def\PY@bc##1{\setlength{\fboxsep}{0pt}\fcolorbox[rgb]{1.00,0.00,0.00}{1,1,1}{\strut ##1}}}
\expandafter\def\csname PY@tok@kc\endcsname{\let\PY@bf=\textbf\def\PY@tc##1{\textcolor[rgb]{0.00,0.50,0.00}{##1}}}
\expandafter\def\csname PY@tok@kd\endcsname{\let\PY@bf=\textbf\def\PY@tc##1{\textcolor[rgb]{0.00,0.50,0.00}{##1}}}
\expandafter\def\csname PY@tok@kn\endcsname{\let\PY@bf=\textbf\def\PY@tc##1{\textcolor[rgb]{0.00,0.50,0.00}{##1}}}
\expandafter\def\csname PY@tok@kr\endcsname{\let\PY@bf=\textbf\def\PY@tc##1{\textcolor[rgb]{0.00,0.50,0.00}{##1}}}
\expandafter\def\csname PY@tok@bp\endcsname{\def\PY@tc##1{\textcolor[rgb]{0.00,0.50,0.00}{##1}}}
\expandafter\def\csname PY@tok@fm\endcsname{\def\PY@tc##1{\textcolor[rgb]{0.00,0.00,1.00}{##1}}}
\expandafter\def\csname PY@tok@vc\endcsname{\def\PY@tc##1{\textcolor[rgb]{0.10,0.09,0.49}{##1}}}
\expandafter\def\csname PY@tok@vg\endcsname{\def\PY@tc##1{\textcolor[rgb]{0.10,0.09,0.49}{##1}}}
\expandafter\def\csname PY@tok@vi\endcsname{\def\PY@tc##1{\textcolor[rgb]{0.10,0.09,0.49}{##1}}}
\expandafter\def\csname PY@tok@vm\endcsname{\def\PY@tc##1{\textcolor[rgb]{0.10,0.09,0.49}{##1}}}
\expandafter\def\csname PY@tok@sa\endcsname{\def\PY@tc##1{\textcolor[rgb]{0.73,0.13,0.13}{##1}}}
\expandafter\def\csname PY@tok@sb\endcsname{\def\PY@tc##1{\textcolor[rgb]{0.73,0.13,0.13}{##1}}}
\expandafter\def\csname PY@tok@sc\endcsname{\def\PY@tc##1{\textcolor[rgb]{0.73,0.13,0.13}{##1}}}
\expandafter\def\csname PY@tok@dl\endcsname{\def\PY@tc##1{\textcolor[rgb]{0.73,0.13,0.13}{##1}}}
\expandafter\def\csname PY@tok@s2\endcsname{\def\PY@tc##1{\textcolor[rgb]{0.73,0.13,0.13}{##1}}}
\expandafter\def\csname PY@tok@sh\endcsname{\def\PY@tc##1{\textcolor[rgb]{0.73,0.13,0.13}{##1}}}
\expandafter\def\csname PY@tok@s1\endcsname{\def\PY@tc##1{\textcolor[rgb]{0.73,0.13,0.13}{##1}}}
\expandafter\def\csname PY@tok@mb\endcsname{\def\PY@tc##1{\textcolor[rgb]{0.40,0.40,0.40}{##1}}}
\expandafter\def\csname PY@tok@mf\endcsname{\def\PY@tc##1{\textcolor[rgb]{0.40,0.40,0.40}{##1}}}
\expandafter\def\csname PY@tok@mh\endcsname{\def\PY@tc##1{\textcolor[rgb]{0.40,0.40,0.40}{##1}}}
\expandafter\def\csname PY@tok@mi\endcsname{\def\PY@tc##1{\textcolor[rgb]{0.40,0.40,0.40}{##1}}}
\expandafter\def\csname PY@tok@il\endcsname{\def\PY@tc##1{\textcolor[rgb]{0.40,0.40,0.40}{##1}}}
\expandafter\def\csname PY@tok@mo\endcsname{\def\PY@tc##1{\textcolor[rgb]{0.40,0.40,0.40}{##1}}}
\expandafter\def\csname PY@tok@ch\endcsname{\let\PY@it=\textit\def\PY@tc##1{\textcolor[rgb]{0.25,0.50,0.50}{##1}}}
\expandafter\def\csname PY@tok@cm\endcsname{\let\PY@it=\textit\def\PY@tc##1{\textcolor[rgb]{0.25,0.50,0.50}{##1}}}
\expandafter\def\csname PY@tok@cpf\endcsname{\let\PY@it=\textit\def\PY@tc##1{\textcolor[rgb]{0.25,0.50,0.50}{##1}}}
\expandafter\def\csname PY@tok@c1\endcsname{\let\PY@it=\textit\def\PY@tc##1{\textcolor[rgb]{0.25,0.50,0.50}{##1}}}
\expandafter\def\csname PY@tok@cs\endcsname{\let\PY@it=\textit\def\PY@tc##1{\textcolor[rgb]{0.25,0.50,0.50}{##1}}}

\def\PYZbs{\char`\\}
\def\PYZus{\char`\_}
\def\PYZob{\char`\{}
\def\PYZcb{\char`\}}
\def\PYZca{\char`\^}
\def\PYZam{\char`\&}
\def\PYZlt{\char`\<}
\def\PYZgt{\char`\>}
\def\PYZsh{\char`\#}
\def\PYZpc{\char`\%}
\def\PYZdl{\char`\$}
\def\PYZhy{\char`\-}
\def\PYZsq{\char`\'}
\def\PYZdq{\char`\"}
\def\PYZti{\char`\~}
% for compatibility with earlier versions
\def\PYZat{@}
\def\PYZlb{[}
\def\PYZrb{]}
\makeatother


    % For linebreaks inside Verbatim environment from package fancyvrb. 
    \makeatletter
        \newbox\Wrappedcontinuationbox 
        \newbox\Wrappedvisiblespacebox 
        \newcommand*\Wrappedvisiblespace {\textcolor{red}{\textvisiblespace}} 
        \newcommand*\Wrappedcontinuationsymbol {\textcolor{red}{\llap{\tiny$\m@th\hookrightarrow$}}} 
        \newcommand*\Wrappedcontinuationindent {3ex } 
        \newcommand*\Wrappedafterbreak {\kern\Wrappedcontinuationindent\copy\Wrappedcontinuationbox} 
        % Take advantage of the already applied Pygments mark-up to insert 
        % potential linebreaks for TeX processing. 
        %        {, <, #, %, $, ' and ": go to next line. 
        %        _, }, ^, &, >, - and ~: stay at end of broken line. 
        % Use of \textquotesingle for straight quote. 
        \newcommand*\Wrappedbreaksatspecials {% 
            \def\PYGZus{\discretionary{\char`\_}{\Wrappedafterbreak}{\char`\_}}% 
            \def\PYGZob{\discretionary{}{\Wrappedafterbreak\char`\{}{\char`\{}}% 
            \def\PYGZcb{\discretionary{\char`\}}{\Wrappedafterbreak}{\char`\}}}% 
            \def\PYGZca{\discretionary{\char`\^}{\Wrappedafterbreak}{\char`\^}}% 
            \def\PYGZam{\discretionary{\char`\&}{\Wrappedafterbreak}{\char`\&}}% 
            \def\PYGZlt{\discretionary{}{\Wrappedafterbreak\char`\<}{\char`\<}}% 
            \def\PYGZgt{\discretionary{\char`\>}{\Wrappedafterbreak}{\char`\>}}% 
            \def\PYGZsh{\discretionary{}{\Wrappedafterbreak\char`\#}{\char`\#}}% 
            \def\PYGZpc{\discretionary{}{\Wrappedafterbreak\char`\%}{\char`\%}}% 
            \def\PYGZdl{\discretionary{}{\Wrappedafterbreak\char`\$}{\char`\$}}% 
            \def\PYGZhy{\discretionary{\char`\-}{\Wrappedafterbreak}{\char`\-}}% 
            \def\PYGZsq{\discretionary{}{\Wrappedafterbreak\textquotesingle}{\textquotesingle}}% 
            \def\PYGZdq{\discretionary{}{\Wrappedafterbreak\char`\"}{\char`\"}}% 
            \def\PYGZti{\discretionary{\char`\~}{\Wrappedafterbreak}{\char`\~}}% 
        } 
        % Some characters . , ; ? ! / are not pygmentized. 
        % This macro makes them "active" and they will insert potential linebreaks 
        \newcommand*\Wrappedbreaksatpunct {% 
            \lccode`\~`\.\lowercase{\def~}{\discretionary{\hbox{\char`\.}}{\Wrappedafterbreak}{\hbox{\char`\.}}}% 
            \lccode`\~`\,\lowercase{\def~}{\discretionary{\hbox{\char`\,}}{\Wrappedafterbreak}{\hbox{\char`\,}}}% 
            \lccode`\~`\;\lowercase{\def~}{\discretionary{\hbox{\char`\;}}{\Wrappedafterbreak}{\hbox{\char`\;}}}% 
            \lccode`\~`\:\lowercase{\def~}{\discretionary{\hbox{\char`\:}}{\Wrappedafterbreak}{\hbox{\char`\:}}}% 
            \lccode`\~`\?\lowercase{\def~}{\discretionary{\hbox{\char`\?}}{\Wrappedafterbreak}{\hbox{\char`\?}}}% 
            \lccode`\~`\!\lowercase{\def~}{\discretionary{\hbox{\char`\!}}{\Wrappedafterbreak}{\hbox{\char`\!}}}% 
            \lccode`\~`\/\lowercase{\def~}{\discretionary{\hbox{\char`\/}}{\Wrappedafterbreak}{\hbox{\char`\/}}}% 
            \catcode`\.\active
            \catcode`\,\active 
            \catcode`\;\active
            \catcode`\:\active
            \catcode`\?\active
            \catcode`\!\active
            \catcode`\/\active 
            \lccode`\~`\~ 	
        }
    \makeatother

    \let\OriginalVerbatim=\Verbatim
    \makeatletter
    \renewcommand{\Verbatim}[1][1]{%
        %\parskip\z@skip
        \sbox\Wrappedcontinuationbox {\Wrappedcontinuationsymbol}%
        \sbox\Wrappedvisiblespacebox {\FV@SetupFont\Wrappedvisiblespace}%
        \def\FancyVerbFormatLine ##1{\hsize\linewidth
            \vtop{\raggedright\hyphenpenalty\z@\exhyphenpenalty\z@
                \doublehyphendemerits\z@\finalhyphendemerits\z@
                \strut ##1\strut}%
        }%
        % If the linebreak is at a space, the latter will be displayed as visible
        % space at end of first line, and a continuation symbol starts next line.
        % Stretch/shrink are however usually zero for typewriter font.
        \def\FV@Space {%
            \nobreak\hskip\z@ plus\fontdimen3\font minus\fontdimen4\font
            \discretionary{\copy\Wrappedvisiblespacebox}{\Wrappedafterbreak}
            {\kern\fontdimen2\font}%
        }%
        
        % Allow breaks at special characters using \PYG... macros.
        \Wrappedbreaksatspecials
        % Breaks at punctuation characters . , ; ? ! and / need catcode=\active 	
        \OriginalVerbatim[#1,codes*=\Wrappedbreaksatpunct]%
    }
    \makeatother

    % Exact colors from NB
    \definecolor{incolor}{HTML}{303F9F}
    \definecolor{outcolor}{HTML}{D84315}
    \definecolor{cellborder}{HTML}{CFCFCF}
    \definecolor{cellbackground}{HTML}{F7F7F7}
    
    % prompt
    \makeatletter
    \newcommand{\boxspacing}{\kern\kvtcb@left@rule\kern\kvtcb@boxsep}
    \makeatother
    \newcommand{\prompt}[4]{
        \ttfamily\llap{{\color{#2}[#3]:\hspace{3pt}#4}}\vspace{-\baselineskip}
    }
    

    
    % Prevent overflowing lines due to hard-to-break entities
    \sloppy 
    % Setup hyperref package
    \hypersetup{
      breaklinks=true,  % so long urls are correctly broken across lines
      colorlinks=true,
      urlcolor=urlcolor,
      linkcolor=linkcolor,
      citecolor=citecolor,
      }
    % Slightly bigger margins than the latex defaults
    
    \geometry{verbose,tmargin=1in,bmargin=1in,lmargin=1in,rmargin=1in}
    
    

\begin{document}
    
    \maketitle
    
    

    
    \hypertarget{introduction}{%
\section{1. Introduction}\label{introduction}}

Nous avons pour projet d'utliser des concepts d'apprentissage fédérés
pour construire un pipeline d'apprentissage automatique basé sur des
appareils mobiles et des appareils informatiques à carte unique tels que
Raspberry .

Youtube est la plus grande plate-forme vidéo au monde avec des millions
d'utilisateurs simultanés au quotidien et une grande influence sur le
comportement, les croyances et les opinions des clients. De ce fait,
l'optimisation des performances vidéo a acquis une valeur économique
tangible et de nombreusesentreprises l'utilisent pour gagner en traction
et susciter l'intérêt pour leurs produits et services.

Dans cette étude, nous visons à créer un modèle pour analyser les effets
des balises vidéo dans les performances d'une vidéo. Nous explorerons à
la fois les caractéristiques quantitatives et qualitatives des balises
vidéo, en les comparant aux performances globales de la vidéo à travers
une série de mesures telles que le nombre de vues, le rapport j'aime /
n'aime pas et le nombre de jours de tendance.

    \begin{tcolorbox}[breakable, size=fbox, boxrule=1pt, pad at break*=1mm,colback=cellbackground, colframe=cellborder]
\prompt{In}{incolor}{38}{\boxspacing}
\begin{Verbatim}[commandchars=\\\{\}]
\PY{k+kn}{import} \PY{n+nn}{pandas} \PY{k}{as} \PY{n+nn}{pd}
\PY{k+kn}{import} \PY{n+nn}{matplotlib}\PY{n+nn}{.}\PY{n+nn}{pyplot} \PY{k}{as} \PY{n+nn}{plt}
\PY{k+kn}{import} \PY{n+nn}{numpy} \PY{k}{as} \PY{n+nn}{np}
\PY{k+kn}{import} \PY{n+nn}{seaborn} \PY{k}{as} \PY{n+nn}{sns}
\PY{k+kn}{import} \PY{n+nn}{json}
\end{Verbatim}
\end{tcolorbox}

    \hypertarget{datasets}{%
\section{2. Datasets}\label{datasets}}

\hypertarget{uxe0-propos-des-donnuxe9es}{%
\subsection{2.1 À propos des données}\label{uxe0-propos-des-donnuxe9es}}

Datasets utilisé: Trending YouTube Video Statistics

L'ensemble de données se compose d'enregistrements quotidiens des 200
vidéos YouTube les plus populaires, classés par pays. Aux fins de cette
analyse, nous nous concentrerons sur les vidéos provenant de France
(FR), Grande-Bretagne (GB), et l'Allemagne (DE) pour la période comprise
entre le 14 novembre 2017 et le 14 juin 2018, soit un total de 205
jours.

Nous avons dans un premier etudié les caractéristiques de chaque pays
indépendamment, puis dans un second temps comparer ceux-ci entre eux.

La description des colonnes pertinentes de l'ensemble de données est
présentée dans le tableau ci-dessous.

    \begin{longtable}[]{@{}llll@{}}
\toprule
\begin{minipage}[b]{0.32\columnwidth}\raggedright
Columns Header\strut
\end{minipage} & \begin{minipage}[b]{0.18\columnwidth}\raggedright
Quantity\strut
\end{minipage} & \begin{minipage}[b]{0.14\columnwidth}\raggedright
Type\strut
\end{minipage} & \begin{minipage}[b]{0.25\columnwidth}\raggedright
Description\strut
\end{minipage}\tabularnewline
\midrule
\endhead
\begin{minipage}[t]{0.32\columnwidth}\raggedright
video\_id\strut
\end{minipage} & \begin{minipage}[t]{0.18\columnwidth}\raggedright
40949\strut
\end{minipage} & \begin{minipage}[t]{0.14\columnwidth}\raggedright
object\strut
\end{minipage} & \begin{minipage}[t]{0.25\columnwidth}\raggedright
Unique Value for each video\strut
\end{minipage}\tabularnewline
\begin{minipage}[t]{0.32\columnwidth}\raggedright
trending\_date\strut
\end{minipage} & \begin{minipage}[t]{0.18\columnwidth}\raggedright
40949\strut
\end{minipage} & \begin{minipage}[t]{0.14\columnwidth}\raggedright
object\strut
\end{minipage} & \begin{minipage}[t]{0.25\columnwidth}\raggedright
Date in which video is trending. Video can trend for over a day.\strut
\end{minipage}\tabularnewline
\begin{minipage}[t]{0.32\columnwidth}\raggedright
title\strut
\end{minipage} & \begin{minipage}[t]{0.18\columnwidth}\raggedright
40949\strut
\end{minipage} & \begin{minipage}[t]{0.14\columnwidth}\raggedright
object\strut
\end{minipage} & \begin{minipage}[t]{0.25\columnwidth}\raggedright
Video's title\strut
\end{minipage}\tabularnewline
\begin{minipage}[t]{0.32\columnwidth}\raggedright
channel\_title\strut
\end{minipage} & \begin{minipage}[t]{0.18\columnwidth}\raggedright
40949\strut
\end{minipage} & \begin{minipage}[t]{0.14\columnwidth}\raggedright
object\strut
\end{minipage} & \begin{minipage}[t]{0.25\columnwidth}\raggedright
Name of uploading channel\strut
\end{minipage}\tabularnewline
\begin{minipage}[t]{0.32\columnwidth}\raggedright
category\_id\strut
\end{minipage} & \begin{minipage}[t]{0.18\columnwidth}\raggedright
40949\strut
\end{minipage} & \begin{minipage}[t]{0.14\columnwidth}\raggedright
int64\strut
\end{minipage} & \begin{minipage}[t]{0.25\columnwidth}\raggedright
Video category\strut
\end{minipage}\tabularnewline
\begin{minipage}[t]{0.32\columnwidth}\raggedright
publish\_time\strut
\end{minipage} & \begin{minipage}[t]{0.18\columnwidth}\raggedright
40949\strut
\end{minipage} & \begin{minipage}[t]{0.14\columnwidth}\raggedright
object\strut
\end{minipage} & \begin{minipage}[t]{0.25\columnwidth}\raggedright
Time of upload\strut
\end{minipage}\tabularnewline
\begin{minipage}[t]{0.32\columnwidth}\raggedright
tags\strut
\end{minipage} & \begin{minipage}[t]{0.18\columnwidth}\raggedright
40949\strut
\end{minipage} & \begin{minipage}[t]{0.14\columnwidth}\raggedright
object\strut
\end{minipage} & \begin{minipage}[t]{0.25\columnwidth}\raggedright
Tags associated with the video\strut
\end{minipage}\tabularnewline
\begin{minipage}[t]{0.32\columnwidth}\raggedright
views\strut
\end{minipage} & \begin{minipage}[t]{0.18\columnwidth}\raggedright
40949\strut
\end{minipage} & \begin{minipage}[t]{0.14\columnwidth}\raggedright
int64\strut
\end{minipage} & \begin{minipage}[t]{0.25\columnwidth}\raggedright
Number of views\strut
\end{minipage}\tabularnewline
\begin{minipage}[t]{0.32\columnwidth}\raggedright
likes\strut
\end{minipage} & \begin{minipage}[t]{0.18\columnwidth}\raggedright
40949\strut
\end{minipage} & \begin{minipage}[t]{0.14\columnwidth}\raggedright
int64\strut
\end{minipage} & \begin{minipage}[t]{0.25\columnwidth}\raggedright
Number of likes\strut
\end{minipage}\tabularnewline
\begin{minipage}[t]{0.32\columnwidth}\raggedright
dislikes\strut
\end{minipage} & \begin{minipage}[t]{0.18\columnwidth}\raggedright
40949\strut
\end{minipage} & \begin{minipage}[t]{0.14\columnwidth}\raggedright
int64\strut
\end{minipage} & \begin{minipage}[t]{0.25\columnwidth}\raggedright
Number of dislikes\strut
\end{minipage}\tabularnewline
\begin{minipage}[t]{0.32\columnwidth}\raggedright
comment\_count\strut
\end{minipage} & \begin{minipage}[t]{0.18\columnwidth}\raggedright
40949\strut
\end{minipage} & \begin{minipage}[t]{0.14\columnwidth}\raggedright
int64\strut
\end{minipage} & \begin{minipage}[t]{0.25\columnwidth}\raggedright
Number of comments\strut
\end{minipage}\tabularnewline
\begin{minipage}[t]{0.32\columnwidth}\raggedright
comments\_disabled\strut
\end{minipage} & \begin{minipage}[t]{0.18\columnwidth}\raggedright
40949\strut
\end{minipage} & \begin{minipage}[t]{0.14\columnwidth}\raggedright
bool\strut
\end{minipage} & \begin{minipage}[t]{0.25\columnwidth}\raggedright
Comment section enabled or not\strut
\end{minipage}\tabularnewline
\begin{minipage}[t]{0.32\columnwidth}\raggedright
ratings\_disabled\strut
\end{minipage} & \begin{minipage}[t]{0.18\columnwidth}\raggedright
40949\strut
\end{minipage} & \begin{minipage}[t]{0.14\columnwidth}\raggedright
bool\strut
\end{minipage} & \begin{minipage}[t]{0.25\columnwidth}\raggedright
Like/Dislike ratio shown\strut
\end{minipage}\tabularnewline
\begin{minipage}[t]{0.32\columnwidth}\raggedright
video\_error\_or\_removed\strut
\end{minipage} & \begin{minipage}[t]{0.18\columnwidth}\raggedright
40949\strut
\end{minipage} & \begin{minipage}[t]{0.14\columnwidth}\raggedright
bool\strut
\end{minipage} & \begin{minipage}[t]{0.25\columnwidth}\raggedright
Was the video deleted or not\strut
\end{minipage}\tabularnewline
\begin{minipage}[t]{0.32\columnwidth}\raggedright
description\strut
\end{minipage} & \begin{minipage}[t]{0.18\columnwidth}\raggedright
40379\strut
\end{minipage} & \begin{minipage}[t]{0.14\columnwidth}\raggedright
object\strut
\end{minipage} & \begin{minipage}[t]{0.25\columnwidth}\raggedright
Video description\strut
\end{minipage}\tabularnewline
\bottomrule
\end{longtable}

    \hypertarget{muxe9thodologie}{%
\section{3. Méthodologie}\label{muxe9thodologie}}

\hypertarget{comprendre-les-performances-viduxe9o}{%
\subsection{3.1 Comprendre les ``performances
vidéo''}\label{comprendre-les-performances-viduxe9o}}

Il existe de nombreuses façons différentes de mesurer les performances
d'une vidéo. Dans notre étude, nous examinerons les performances vidéo
sur la base de 2 aspects clés: - \textbf{1. L'aspect de Popularité} Nous
etudirins ici les indicateurs relevant : - (Nombre de) J'aime - (Nombre
de) commentaire

\begin{itemize}
\tightlist
\item
  \textbf{2. L'Engagement}

  \begin{itemize}
  \tightlist
  \item
    (Nombre de) vues
  \item
    durée de la tendance = (date actuelle - date\_tendance)
  \end{itemize}
\end{itemize}

    \hypertarget{laspect-de-popularituxe9}{%
\subsubsection{3.1.1 L'aspect de
Popularité}\label{laspect-de-popularituxe9}}

Tout d'habors, nous avons pris connaissance du dataset en explorant les
idicateurs relevants le nombre de j'aime et de commentaire en plotant
nos premiers histograms.

    \begin{tcolorbox}[breakable, size=fbox, boxrule=1pt, pad at break*=1mm,colback=cellbackground, colframe=cellborder]
\prompt{In}{incolor}{39}{\boxspacing}
\begin{Verbatim}[commandchars=\\\{\}]
\PY{n}{videos} \PY{o}{=} \PY{n}{pd}\PY{o}{.}\PY{n}{read\PYZus{}csv}\PY{p}{(}\PY{l+s+s2}{\PYZdq{}}\PY{l+s+s2}{FRvideos.csv}\PY{l+s+s2}{\PYZdq{}}\PY{p}{)}

\PY{n}{plt}\PY{o}{.}\PY{n}{rc}\PY{p}{(}\PY{l+s+s1}{\PYZsq{}}\PY{l+s+s1}{figure.subplot}\PY{l+s+s1}{\PYZsq{}}\PY{p}{,} \PY{n}{wspace}\PY{o}{=}\PY{l+m+mf}{0.9}\PY{p}{)}
\PY{n}{fig}\PY{p}{,} \PY{n}{ax} \PY{o}{=} \PY{n}{plt}\PY{o}{.}\PY{n}{subplots}\PY{p}{(}\PY{p}{)}
\PY{n}{\PYZus{}} \PY{o}{=} \PY{n}{sns}\PY{o}{.}\PY{n}{distplot}\PY{p}{(}\PY{n}{videos}\PY{p}{[}\PY{l+s+s2}{\PYZdq{}}\PY{l+s+s2}{likes}\PY{l+s+s2}{\PYZdq{}}\PY{p}{]}\PY{p}{,} \PY{n}{kde}\PY{o}{=}\PY{k+kc}{False}\PY{p}{,} \PY{n}{hist\PYZus{}kws}\PY{o}{=}\PY{p}{\PYZob{}}\PY{l+s+s1}{\PYZsq{}}\PY{l+s+s1}{alpha}\PY{l+s+s1}{\PYZsq{}}\PY{p}{:} \PY{l+m+mi}{1}\PY{p}{\PYZcb{}}\PY{p}{,} 
                 \PY{n}{bins}\PY{o}{=}\PY{n}{np}\PY{o}{.}\PY{n}{linspace}\PY{p}{(}\PY{l+m+mi}{0}\PY{p}{,} \PY{l+m+mf}{6e6}\PY{p}{,} \PY{l+m+mi}{61}\PY{p}{)}\PY{p}{,} \PY{n}{ax}\PY{o}{=}\PY{n}{ax}\PY{p}{)}
\PY{n}{\PYZus{}} \PY{o}{=} \PY{n}{ax}\PY{o}{.}\PY{n}{set}\PY{p}{(}\PY{n}{xlabel}\PY{o}{=}\PY{l+s+s2}{\PYZdq{}}\PY{l+s+s2}{Likes}\PY{l+s+s2}{\PYZdq{}}\PY{p}{,} \PY{n}{ylabel}\PY{o}{=}\PY{l+s+s2}{\PYZdq{}}\PY{l+s+s2}{No. of videos}\PY{l+s+s2}{\PYZdq{}}\PY{p}{)}
\PY{n}{\PYZus{}} \PY{o}{=} \PY{n}{plt}\PY{o}{.}\PY{n}{xticks}\PY{p}{(}\PY{n}{rotation}\PY{o}{=}\PY{l+m+mi}{90}\PY{p}{)}
\end{Verbatim}
\end{tcolorbox}

    \begin{center}
    \adjustimage{max size={0.9\linewidth}{0.9\paperheight}}{projet_files/projet_6_0.png}
    \end{center}
    { \hspace*{\fill} \\}
    
    Nous notons que la grande majorité des vidéos tendances ont entre 0 et 1
000 000 likes. Laissez-nous tracer l'histogramme uniquement pour les
vidéos avec 100 000 likes ou moin s pour voir de plus près la
distribution des données

    \begin{tcolorbox}[breakable, size=fbox, boxrule=1pt, pad at break*=1mm,colback=cellbackground, colframe=cellborder]
\prompt{In}{incolor}{40}{\boxspacing}
\begin{Verbatim}[commandchars=\\\{\}]
\PY{n}{fig}\PY{p}{,} \PY{n}{ax} \PY{o}{=} \PY{n}{plt}\PY{o}{.}\PY{n}{subplots}\PY{p}{(}\PY{p}{)}
\PY{n}{\PYZus{}} \PY{o}{=} \PY{n}{sns}\PY{o}{.}\PY{n}{distplot}\PY{p}{(}\PY{n}{videos}\PY{p}{[}\PY{n}{videos}\PY{p}{[}\PY{l+s+s2}{\PYZdq{}}\PY{l+s+s2}{likes}\PY{l+s+s2}{\PYZdq{}}\PY{p}{]} \PY{o}{\PYZlt{}}\PY{o}{=} \PY{l+m+mf}{1e5}\PY{p}{]}\PY{p}{[}\PY{l+s+s2}{\PYZdq{}}\PY{l+s+s2}{likes}\PY{l+s+s2}{\PYZdq{}}\PY{p}{]}\PY{p}{,} \PY{n}{kde}\PY{o}{=}\PY{k+kc}{False}\PY{p}{,} \PY{n}{hist\PYZus{}kws}\PY{o}{=}\PY{p}{\PYZob{}}\PY{l+s+s1}{\PYZsq{}}\PY{l+s+s1}{alpha}\PY{l+s+s1}{\PYZsq{}}\PY{p}{:} \PY{l+m+mi}{1}\PY{p}{\PYZcb{}}\PY{p}{,} \PY{n}{ax}\PY{o}{=}\PY{n}{ax}\PY{p}{)}
\PY{n}{\PYZus{}} \PY{o}{=} \PY{n}{ax}\PY{o}{.}\PY{n}{set}\PY{p}{(}\PY{n}{xlabel}\PY{o}{=}\PY{l+s+s2}{\PYZdq{}}\PY{l+s+s2}{Likes}\PY{l+s+s2}{\PYZdq{}}\PY{p}{,} \PY{n}{ylabel}\PY{o}{=}\PY{l+s+s2}{\PYZdq{}}\PY{l+s+s2}{No. of videos}\PY{l+s+s2}{\PYZdq{}}\PY{p}{)}
\end{Verbatim}
\end{tcolorbox}

    \begin{center}
    \adjustimage{max size={0.9\linewidth}{0.9\paperheight}}{projet_files/projet_8_0.png}
    \end{center}
    { \hspace*{\fill} \\}
    
    Maintenant, nous pouvons voir que la majorité des vidéos tendances ont
40000 likes ou moins avec un pic pour les vidéos avec 2000 likes ou
moins.

Voyons le pourcentage exact de vidéos avec moins de 40000 likes

    \begin{tcolorbox}[breakable, size=fbox, boxrule=1pt, pad at break*=1mm,colback=cellbackground, colframe=cellborder]
\prompt{In}{incolor}{41}{\boxspacing}
\begin{Verbatim}[commandchars=\\\{\}]
\PY{n}{videos}\PY{p}{[}\PY{n}{videos}\PY{p}{[}\PY{l+s+s1}{\PYZsq{}}\PY{l+s+s1}{likes}\PY{l+s+s1}{\PYZsq{}}\PY{p}{]} \PY{o}{\PYZlt{}} \PY{l+m+mf}{4e4}\PY{p}{]}\PY{p}{[}\PY{l+s+s1}{\PYZsq{}}\PY{l+s+s1}{likes}\PY{l+s+s1}{\PYZsq{}}\PY{p}{]}\PY{o}{.}\PY{n}{count}\PY{p}{(}\PY{p}{)} \PY{o}{/} \PY{n}{videos}\PY{p}{[}\PY{l+s+s1}{\PYZsq{}}\PY{l+s+s1}{likes}\PY{l+s+s1}{\PYZsq{}}\PY{p}{]}\PY{o}{.}\PY{n}{count}\PY{p}{(}\PY{p}{)} \PY{o}{*} \PY{l+m+mi}{100}
\end{Verbatim}
\end{tcolorbox}

            \begin{tcolorbox}[breakable, size=fbox, boxrule=.5pt, pad at break*=1mm, opacityfill=0]
\prompt{Out}{outcolor}{41}{\boxspacing}
\begin{Verbatim}[commandchars=\\\{\}]
92.06119241724781
\end{Verbatim}
\end{tcolorbox}
        
    De même, nous pouvons voir que le pourcentage de vidéos avec moins de
100 000 likes est d'environ 92\%

    \begin{tcolorbox}[breakable, size=fbox, boxrule=1pt, pad at break*=1mm,colback=cellbackground, colframe=cellborder]
\prompt{In}{incolor}{42}{\boxspacing}
\begin{Verbatim}[commandchars=\\\{\}]
\PY{n}{fig}\PY{p}{,} \PY{n}{ax} \PY{o}{=} \PY{n}{plt}\PY{o}{.}\PY{n}{subplots}\PY{p}{(}\PY{p}{)}
\PY{n}{\PYZus{}} \PY{o}{=} \PY{n}{sns}\PY{o}{.}\PY{n}{distplot}\PY{p}{(}\PY{n}{videos}\PY{p}{[}\PY{n}{videos}\PY{p}{[}\PY{l+s+s2}{\PYZdq{}}\PY{l+s+s2}{comment\PYZus{}count}\PY{l+s+s2}{\PYZdq{}}\PY{p}{]} \PY{o}{\PYZlt{}} \PY{l+m+mi}{200000}\PY{p}{]}\PY{p}{[}\PY{l+s+s2}{\PYZdq{}}\PY{l+s+s2}{comment\PYZus{}count}\PY{l+s+s2}{\PYZdq{}}\PY{p}{]}\PY{p}{,} \PY{n}{kde}\PY{o}{=}\PY{k+kc}{False}\PY{p}{,} \PY{n}{rug}\PY{o}{=}\PY{k+kc}{False}\PY{p}{,} \PY{n}{hist\PYZus{}kws}\PY{o}{=}\PY{p}{\PYZob{}}\PY{l+s+s1}{\PYZsq{}}\PY{l+s+s1}{alpha}\PY{l+s+s1}{\PYZsq{}}\PY{p}{:} \PY{l+m+mi}{1}\PY{p}{\PYZcb{}}\PY{p}{,} 
                 \PY{n}{bins}\PY{o}{=}\PY{n}{np}\PY{o}{.}\PY{n}{linspace}\PY{p}{(}\PY{l+m+mi}{0}\PY{p}{,} \PY{l+m+mf}{2e5}\PY{p}{,} \PY{l+m+mi}{49}\PY{p}{)}\PY{p}{,} \PY{n}{ax}\PY{o}{=}\PY{n}{ax}\PY{p}{)}
\PY{n}{\PYZus{}} \PY{o}{=} \PY{n}{ax}\PY{o}{.}\PY{n}{set}\PY{p}{(}\PY{n}{xlabel}\PY{o}{=}\PY{l+s+s2}{\PYZdq{}}\PY{l+s+s2}{Comment Count}\PY{l+s+s2}{\PYZdq{}}\PY{p}{,} \PY{n}{ylabel}\PY{o}{=}\PY{l+s+s2}{\PYZdq{}}\PY{l+s+s2}{No. of videos}\PY{l+s+s2}{\PYZdq{}}\PY{p}{)}
\end{Verbatim}
\end{tcolorbox}

    \begin{center}
    \adjustimage{max size={0.9\linewidth}{0.9\paperheight}}{projet_files/projet_12_0.png}
    \end{center}
    { \hspace*{\fill} \\}
    
    \begin{tcolorbox}[breakable, size=fbox, boxrule=1pt, pad at break*=1mm,colback=cellbackground, colframe=cellborder]
\prompt{In}{incolor}{43}{\boxspacing}
\begin{Verbatim}[commandchars=\\\{\}]
\PY{n}{videos}\PY{p}{[}\PY{n}{videos}\PY{p}{[}\PY{l+s+s1}{\PYZsq{}}\PY{l+s+s1}{likes}\PY{l+s+s1}{\PYZsq{}}\PY{p}{]} \PY{o}{\PYZlt{}} \PY{l+m+mi}{4000}\PY{p}{]}\PY{p}{[}\PY{l+s+s1}{\PYZsq{}}\PY{l+s+s1}{likes}\PY{l+s+s1}{\PYZsq{}}\PY{p}{]}\PY{o}{.}\PY{n}{count}\PY{p}{(}\PY{p}{)} \PY{o}{/} \PY{n}{videos}\PY{p}{[}\PY{l+s+s1}{\PYZsq{}}\PY{l+s+s1}{likes}\PY{l+s+s1}{\PYZsq{}}\PY{p}{]}\PY{o}{.}\PY{n}{count}\PY{p}{(}\PY{p}{)} \PY{o}{*} \PY{l+m+mi}{100}
\end{Verbatim}
\end{tcolorbox}

            \begin{tcolorbox}[breakable, size=fbox, boxrule=.5pt, pad at break*=1mm, opacityfill=0]
\prompt{Out}{outcolor}{43}{\boxspacing}
\begin{Verbatim}[commandchars=\\\{\}]
63.75110499950889
\end{Verbatim}
\end{tcolorbox}
        
    De même, nous pouvons voir que le pourcentage de commentaire avec moins
de 4000 likes est d'environ 63\%

    \hypertarget{lengagement}{%
\subsubsection{3.1.2 L'Engagement}\label{lengagement}}

Dans cette partie, nous etudions: - (Nombre de) vues - durée de la
tendance = (date actuelle - date\_tendance)

Avant l'algorithme réel, nous devons sélectionner nos données d'entrée.

    Tout d'abord, les vidéos qui ne sont pas utiles pour notre cas sont
supprimées de l'échantillon. Cela inclut les vidéos avec des erreurs, où
les commentaires et / ou les évaluations sont désactivés et les vidéos
en double.

Par la suite, nous avons ajouter deux colonnes : ``publish\_time'' et
``trending\_date''. Nous les avons reformaté pour pouvoir par la suite
etudier plus facilement la correlation entre les dates de sorties des
videos, et leur appartions en tendance.

    \begin{tcolorbox}[breakable, size=fbox, boxrule=1pt, pad at break*=1mm,colback=cellbackground, colframe=cellborder]
\prompt{In}{incolor}{44}{\boxspacing}
\begin{Verbatim}[commandchars=\\\{\}]
\PY{c+c1}{\PYZsh{} count days of trending}
\PY{c+c1}{\PYZsh{} trending\PYZus{}period = videos\PYZus{}clean.groupby([\PYZsq{}video\PYZus{}id\PYZsq{}]).size().reset\PYZus{}index(name=\PYZsq{}trending\PYZus{}period\PYZsq{})}

\PY{n}{videos\PYZus{}clean} \PY{o}{=} \PY{n}{videos}\PY{o}{.}\PY{n}{drop}\PY{p}{(}\PY{n}{columns}\PY{o}{=}\PY{p}{[}\PY{l+s+s1}{\PYZsq{}}\PY{l+s+s1}{thumbnail\PYZus{}link}\PY{l+s+s1}{\PYZsq{}}\PY{p}{,} \PY{l+s+s1}{\PYZsq{}}\PY{l+s+s1}{comments\PYZus{}disabled}\PY{l+s+s1}{\PYZsq{}}\PY{p}{,} \PY{l+s+s1}{\PYZsq{}}\PY{l+s+s1}{ratings\PYZus{}disabled}\PY{l+s+s1}{\PYZsq{}}\PY{p}{,} \PY{l+s+s1}{\PYZsq{}}\PY{l+s+s1}{video\PYZus{}error\PYZus{}or\PYZus{}removed}\PY{l+s+s1}{\PYZsq{}}\PY{p}{,} \PY{l+s+s1}{\PYZsq{}}\PY{l+s+s1}{description}\PY{l+s+s1}{\PYZsq{}}\PY{p}{,} \PY{l+s+s1}{\PYZsq{}}\PY{l+s+s1}{publish\PYZus{}time}\PY{l+s+s1}{\PYZsq{}}\PY{p}{,} \PY{l+s+s1}{\PYZsq{}}\PY{l+s+s1}{trending\PYZus{}date}\PY{l+s+s1}{\PYZsq{}}\PY{p}{]}\PY{p}{)}

\PY{c+c1}{\PYZsh{} formating date}
\PY{n}{videos\PYZus{}clean}\PY{p}{[}\PY{l+s+s1}{\PYZsq{}}\PY{l+s+s1}{publish\PYZus{}time}\PY{l+s+s1}{\PYZsq{}}\PY{p}{]} \PY{o}{=} \PY{n}{pd}\PY{o}{.}\PY{n}{to\PYZus{}datetime}\PY{p}{(}\PY{n}{videos}\PY{p}{[}\PY{l+s+s1}{\PYZsq{}}\PY{l+s+s1}{publish\PYZus{}time}\PY{l+s+s1}{\PYZsq{}}\PY{p}{]}\PY{p}{,} \PY{n+nb}{format}\PY{o}{=}\PY{l+s+s1}{\PYZsq{}}\PY{l+s+s1}{\PYZpc{}}\PY{l+s+s1}{Y\PYZhy{}}\PY{l+s+s1}{\PYZpc{}}\PY{l+s+s1}{m\PYZhy{}}\PY{l+s+si}{\PYZpc{}d}\PY{l+s+s1}{T}\PY{l+s+s1}{\PYZpc{}}\PY{l+s+s1}{H:}\PY{l+s+s1}{\PYZpc{}}\PY{l+s+s1}{M:}\PY{l+s+s1}{\PYZpc{}}\PY{l+s+s1}{S.000Z}\PY{l+s+s1}{\PYZsq{}}\PY{p}{,} \PY{n}{utc}\PY{o}{=}\PY{k+kc}{False}\PY{p}{)}
\PY{n}{videos\PYZus{}clean}\PY{p}{[}\PY{l+s+s1}{\PYZsq{}}\PY{l+s+s1}{trending\PYZus{}date}\PY{l+s+s1}{\PYZsq{}}\PY{p}{]} \PY{o}{=} \PY{n}{pd}\PY{o}{.}\PY{n}{to\PYZus{}datetime}\PY{p}{(}\PY{n}{videos}\PY{p}{[}\PY{l+s+s1}{\PYZsq{}}\PY{l+s+s1}{trending\PYZus{}date}\PY{l+s+s1}{\PYZsq{}}\PY{p}{]}\PY{p}{,} \PY{n+nb}{format}\PY{o}{=}\PY{l+s+s1}{\PYZsq{}}\PY{l+s+s1}{\PYZpc{}}\PY{l+s+s1}{y.}\PY{l+s+si}{\PYZpc{}d}\PY{l+s+s1}{.}\PY{l+s+s1}{\PYZpc{}}\PY{l+s+s1}{m}\PY{l+s+s1}{\PYZsq{}}\PY{p}{,} \PY{n}{utc}\PY{o}{=}\PY{k+kc}{False}\PY{p}{)}

\PY{c+c1}{\PYZsh{} drop duplicate and keep newest videos}
\PY{n}{videos\PYZus{}clean} \PY{o}{=} \PY{n}{videos\PYZus{}clean}\PY{o}{.}\PY{n}{sort\PYZus{}values}\PY{p}{(}\PY{n}{by}\PY{o}{=}\PY{p}{[}\PY{l+s+s1}{\PYZsq{}}\PY{l+s+s1}{trending\PYZus{}date}\PY{l+s+s1}{\PYZsq{}}\PY{p}{]}\PY{p}{)}\PY{o}{.}\PY{n}{reset\PYZus{}index}\PY{p}{(}\PY{p}{)}
\end{Verbatim}
\end{tcolorbox}

    \begin{tcolorbox}[breakable, size=fbox, boxrule=1pt, pad at break*=1mm,colback=cellbackground, colframe=cellborder]
\prompt{In}{incolor}{45}{\boxspacing}
\begin{Verbatim}[commandchars=\\\{\}]
\PY{n}{views\PYZus{}trending} \PY{o}{=} \PY{n}{videos\PYZus{}clean}\PY{o}{.}\PY{n}{filter}\PY{p}{(}\PY{p}{[}\PY{l+s+s1}{\PYZsq{}}\PY{l+s+s1}{video\PYZus{}id}\PY{l+s+s1}{\PYZsq{}}\PY{p}{,} \PY{l+s+s1}{\PYZsq{}}\PY{l+s+s1}{views}\PY{l+s+s1}{\PYZsq{}}\PY{p}{,} \PY{l+s+s1}{\PYZsq{}}\PY{l+s+s1}{trending\PYZus{}date}\PY{l+s+s1}{\PYZsq{}}\PY{p}{]}\PY{p}{)}
\PY{n}{views\PYZus{}trending\PYZus{}grouped} \PY{o}{=} \PY{n}{views\PYZus{}trending}\PY{o}{.}\PY{n}{groupby}\PY{p}{(}\PY{n}{by}\PY{o}{=}\PY{l+s+s1}{\PYZsq{}}\PY{l+s+s1}{video\PYZus{}id}\PY{l+s+s1}{\PYZsq{}}\PY{p}{)}

\PY{n}{views\PYZus{}increase} \PY{o}{=} \PY{n}{pd}\PY{o}{.}\PY{n}{DataFrame}\PY{p}{(}\PY{n}{views\PYZus{}trending\PYZus{}grouped}\PY{p}{[}\PY{l+s+s1}{\PYZsq{}}\PY{l+s+s1}{views}\PY{l+s+s1}{\PYZsq{}}\PY{p}{]}\PY{o}{.}\PY{n}{count}\PY{p}{(}\PY{p}{)}\PY{p}{)}

\PY{n}{views\PYZus{}diff} \PY{o}{=} \PY{p}{[}\PY{p}{]}
\PY{k}{for} \PY{n}{video\PYZus{}id} \PY{o+ow}{in} \PY{n}{views\PYZus{}trending\PYZus{}grouped}\PY{o}{.}\PY{n}{groups}\PY{p}{:}
    \PY{n}{video\PYZus{}group} \PY{o}{=} \PY{n}{views\PYZus{}trending\PYZus{}grouped}\PY{o}{.}\PY{n}{get\PYZus{}group}\PY{p}{(}\PY{n}{video\PYZus{}id}\PY{p}{)}
    \PY{n}{views\PYZus{}diff}\PY{o}{.}\PY{n}{append}\PY{p}{(}\PY{n}{video\PYZus{}group}\PY{p}{[}\PY{l+s+s1}{\PYZsq{}}\PY{l+s+s1}{views}\PY{l+s+s1}{\PYZsq{}}\PY{p}{]}\PY{o}{.}\PY{n}{max}\PY{p}{(}\PY{p}{)} \PY{o}{\PYZhy{}} \PY{n}{video\PYZus{}group}\PY{p}{[}\PY{l+s+s1}{\PYZsq{}}\PY{l+s+s1}{views}\PY{l+s+s1}{\PYZsq{}}\PY{p}{]}\PY{o}{.}\PY{n}{min}\PY{p}{(}\PY{p}{)}\PY{p}{)}
\PY{n}{views\PYZus{}increase}\PY{p}{[}\PY{l+s+s1}{\PYZsq{}}\PY{l+s+s1}{views\PYZus{}diff}\PY{l+s+s1}{\PYZsq{}}\PY{p}{]} \PY{o}{=} \PY{n}{views\PYZus{}diff}
\end{Verbatim}
\end{tcolorbox}

    Par la suite, nous avons selectionné uniquement les videos qui sont
restées en tendance plus d'un jour. Nous avons vu ainsi evaluer le
nombre de vue gagné pour les videos.

    \begin{tcolorbox}[breakable, size=fbox, boxrule=1pt, pad at break*=1mm,colback=cellbackground, colframe=cellborder]
\prompt{In}{incolor}{46}{\boxspacing}
\begin{Verbatim}[commandchars=\\\{\}]
\PY{n}{views\PYZus{}increase\PYZus{}multi\PYZus{}days} \PY{o}{=} \PY{n}{views\PYZus{}increase}\PY{p}{[}\PY{n}{views\PYZus{}increase}\PY{p}{[}\PY{l+s+s2}{\PYZdq{}}\PY{l+s+s2}{views}\PY{l+s+s2}{\PYZdq{}}\PY{p}{]} \PY{o}{!=} \PY{l+m+mi}{1}\PY{p}{]}

\PY{n}{views\PYZus{}increase\PYZus{}multi\PYZus{}days}\PY{p}{[}\PY{l+s+s1}{\PYZsq{}}\PY{l+s+s1}{views\PYZus{}diff\PYZus{}per\PYZus{}day}\PY{l+s+s1}{\PYZsq{}}\PY{p}{]} \PY{o}{=} \PY{n}{views\PYZus{}increase\PYZus{}multi\PYZus{}days}\PY{p}{[}\PY{l+s+s1}{\PYZsq{}}\PY{l+s+s1}{views\PYZus{}diff}\PY{l+s+s1}{\PYZsq{}}\PY{p}{]} \PY{o}{/} \PY{n}{views\PYZus{}increase\PYZus{}multi\PYZus{}days}\PY{p}{[}\PY{l+s+s1}{\PYZsq{}}\PY{l+s+s1}{views}\PY{l+s+s1}{\PYZsq{}}\PY{p}{]}

\PY{n}{nb\PYZus{}vid} \PY{o}{=} \PY{n+nb}{len}\PY{p}{(}\PY{n}{views\PYZus{}increase\PYZus{}multi\PYZus{}days}\PY{p}{)}
\PY{n}{percentage} \PY{o}{=} \PY{l+m+mi}{1}\PY{o}{/}\PY{l+m+mi}{100}
\PY{n}{keeped\PYZus{}video} \PY{o}{=} \PY{n+nb}{int}\PY{p}{(}\PY{n}{nb\PYZus{}vid}\PY{o}{*}\PY{n}{percentage}\PY{p}{)}

\PY{n}{views\PYZus{}increase\PYZus{}multi\PYZus{}days}\PY{o}{.}\PY{n}{describe}\PY{p}{(}\PY{p}{)}
\PY{n}{views\PYZus{}increase\PYZus{}multi\PYZus{}days}\PY{o}{.}\PY{n}{sort\PYZus{}values}\PY{p}{(}\PY{l+s+s1}{\PYZsq{}}\PY{l+s+s1}{views\PYZus{}diff\PYZus{}per\PYZus{}day}\PY{l+s+s1}{\PYZsq{}}\PY{p}{)}

\PY{n}{view} \PY{o}{=} \PY{n}{views\PYZus{}increase\PYZus{}multi\PYZus{}days}\PY{o}{.}\PY{n}{sort\PYZus{}values}\PY{p}{(}\PY{l+s+s1}{\PYZsq{}}\PY{l+s+s1}{views\PYZus{}diff\PYZus{}per\PYZus{}day}\PY{l+s+s1}{\PYZsq{}}\PY{p}{)}\PY{p}{[}\PY{p}{:}\PY{n}{keeped\PYZus{}video}\PY{p}{]}
\PY{n}{view}\PY{o}{.}\PY{n}{plot}\PY{p}{(}\PY{n}{y}\PY{o}{=}\PY{l+s+s1}{\PYZsq{}}\PY{l+s+s1}{views\PYZus{}diff\PYZus{}per\PYZus{}day}\PY{l+s+s1}{\PYZsq{}}\PY{p}{,} \PY{n}{figsize}\PY{o}{=}\PY{p}{(}\PY{l+m+mi}{10}\PY{p}{,}\PY{l+m+mi}{7}\PY{p}{)}\PY{p}{)}
\PY{n}{plt}\PY{o}{.}\PY{n}{title}\PY{p}{(}\PY{l+s+s2}{\PYZdq{}}\PY{l+s+s2}{Gain de vue par journée en tendance}\PY{l+s+s2}{\PYZdq{}}\PY{p}{)}
\PY{n}{plt}\PY{o}{.}\PY{n}{xlabel}\PY{p}{(}\PY{l+s+s2}{\PYZdq{}}\PY{l+s+s2}{Vidéos en tendance plusieur jour}\PY{l+s+s2}{\PYZdq{}}\PY{p}{)}
\PY{n}{plt}\PY{o}{.}\PY{n}{ylabel}\PY{p}{(}\PY{l+s+s2}{\PYZdq{}}\PY{l+s+s2}{Nombre de vue}\PY{l+s+s2}{\PYZdq{}}\PY{p}{)}
\PY{n}{plt}\PY{o}{.}\PY{n}{show}\PY{p}{(}\PY{p}{)}
\end{Verbatim}
\end{tcolorbox}

    \begin{Verbatim}[commandchars=\\\{\}]
c:\textbackslash{}python37\textbackslash{}lib\textbackslash{}site-packages\textbackslash{}ipykernel\_launcher.py:3: SettingWithCopyWarning:
A value is trying to be set on a copy of a slice from a DataFrame.
Try using .loc[row\_indexer,col\_indexer] = value instead

See the caveats in the documentation: https://pandas.pydata.org/pandas-
docs/stable/user\_guide/indexing.html\#returning-a-view-versus-a-copy
  This is separate from the ipykernel package so we can avoid doing imports
until
    \end{Verbatim}

    \begin{center}
    \adjustimage{max size={0.9\linewidth}{0.9\paperheight}}{projet_files/projet_20_1.png}
    \end{center}
    { \hspace*{\fill} \\}
    
    \hypertarget{conclusion-2.engagement}{%
\section{\textgreater\textgreater\textgreater{} CONCLUSION 2.Engagement
\textless\textless\textless{}}\label{conclusion-2.engagement}}

    \hypertarget{impact-de-la-date-de-parution}{%
\subsection{3.2 Impact de la date de
parution}\label{impact-de-la-date-de-parution}}

\hypertarget{objectif}{%
\subsubsection{3.2.1 Objectif}\label{objectif}}

Nous nous somme posé la question si le jour de parution de la vidéo
pouvais optimiser nos chance de rentrer en tendance.

L'objectif de cette analyse est savoir qu'elle jour est le plus propice
pour poster une vidéo. On peut essayer de chercher des indice en
mesurant pour chaque video la difference entre son jour de publication
et celui de son entré en tendance. Puis des les afficher par jour. On
pourrait alors par la suite essayer de regarder par category pour voir
si certaine catégorie ce demarque de la tendance génerale.

    \begin{tcolorbox}[breakable, size=fbox, boxrule=1pt, pad at break*=1mm,colback=cellbackground, colframe=cellborder]
\prompt{In}{incolor}{47}{\boxspacing}
\begin{Verbatim}[commandchars=\\\{\}]
\PY{k+kn}{import} \PY{n+nn}{pandas} \PY{k}{as} \PY{n+nn}{pd}
\PY{k+kn}{import} \PY{n+nn}{matplotlib}\PY{n+nn}{.}\PY{n+nn}{pyplot} \PY{k}{as} \PY{n+nn}{plt}
\PY{k+kn}{import} \PY{n+nn}{numpy} \PY{k}{as} \PY{n+nn}{np}

\PY{k}{def} \PY{n+nf}{clean\PYZus{}youtube\PYZus{}dataset}\PY{p}{(}\PY{n}{csv\PYZus{}file}\PY{p}{,} \PY{n}{keeped}\PY{p}{)}\PY{p}{:}
    \PY{n}{videos} \PY{o}{=} \PY{n}{pd}\PY{o}{.}\PY{n}{read\PYZus{}csv}\PY{p}{(}\PY{n}{csv\PYZus{}file}\PY{p}{)}

    \PY{n}{videos\PYZus{}clean} \PY{o}{=} \PY{n}{videos}\PY{o}{.}\PY{n}{drop}\PY{p}{(}\PY{n}{columns}\PY{o}{=}\PY{p}{[}\PY{l+s+s1}{\PYZsq{}}\PY{l+s+s1}{thumbnail\PYZus{}link}\PY{l+s+s1}{\PYZsq{}}\PY{p}{,} \PY{l+s+s1}{\PYZsq{}}\PY{l+s+s1}{comments\PYZus{}disabled}\PY{l+s+s1}{\PYZsq{}}\PY{p}{,} \PY{l+s+s1}{\PYZsq{}}\PY{l+s+s1}{ratings\PYZus{}disabled}\PY{l+s+s1}{\PYZsq{}}\PY{p}{,} \PY{l+s+s1}{\PYZsq{}}\PY{l+s+s1}{video\PYZus{}error\PYZus{}or\PYZus{}removed}\PY{l+s+s1}{\PYZsq{}}\PY{p}{,} \PY{l+s+s1}{\PYZsq{}}\PY{l+s+s1}{description}\PY{l+s+s1}{\PYZsq{}}\PY{p}{,} \PY{l+s+s1}{\PYZsq{}}\PY{l+s+s1}{publish\PYZus{}time}\PY{l+s+s1}{\PYZsq{}}\PY{p}{,} \PY{l+s+s1}{\PYZsq{}}\PY{l+s+s1}{trending\PYZus{}date}\PY{l+s+s1}{\PYZsq{}}\PY{p}{]}\PY{p}{)}

    \PY{c+c1}{\PYZsh{} formating date}
    \PY{n}{videos\PYZus{}clean}\PY{p}{[}\PY{l+s+s1}{\PYZsq{}}\PY{l+s+s1}{publish\PYZus{}time}\PY{l+s+s1}{\PYZsq{}}\PY{p}{]} \PY{o}{=} \PY{n}{pd}\PY{o}{.}\PY{n}{to\PYZus{}datetime}\PY{p}{(}\PY{n}{videos}\PY{p}{[}\PY{l+s+s1}{\PYZsq{}}\PY{l+s+s1}{publish\PYZus{}time}\PY{l+s+s1}{\PYZsq{}}\PY{p}{]}\PY{p}{,} \PY{n+nb}{format}\PY{o}{=}\PY{l+s+s1}{\PYZsq{}}\PY{l+s+s1}{\PYZpc{}}\PY{l+s+s1}{Y\PYZhy{}}\PY{l+s+s1}{\PYZpc{}}\PY{l+s+s1}{m\PYZhy{}}\PY{l+s+si}{\PYZpc{}d}\PY{l+s+s1}{T}\PY{l+s+s1}{\PYZpc{}}\PY{l+s+s1}{H:}\PY{l+s+s1}{\PYZpc{}}\PY{l+s+s1}{M:}\PY{l+s+s1}{\PYZpc{}}\PY{l+s+s1}{S.000Z}\PY{l+s+s1}{\PYZsq{}}\PY{p}{,} \PY{n}{utc}\PY{o}{=}\PY{k+kc}{False}\PY{p}{)}
    \PY{n}{videos\PYZus{}clean}\PY{p}{[}\PY{l+s+s1}{\PYZsq{}}\PY{l+s+s1}{trending\PYZus{}date}\PY{l+s+s1}{\PYZsq{}}\PY{p}{]} \PY{o}{=} \PY{n}{pd}\PY{o}{.}\PY{n}{to\PYZus{}datetime}\PY{p}{(}\PY{n}{videos}\PY{p}{[}\PY{l+s+s1}{\PYZsq{}}\PY{l+s+s1}{trending\PYZus{}date}\PY{l+s+s1}{\PYZsq{}}\PY{p}{]}\PY{p}{,} \PY{n+nb}{format}\PY{o}{=}\PY{l+s+s1}{\PYZsq{}}\PY{l+s+s1}{\PYZpc{}}\PY{l+s+s1}{y.}\PY{l+s+si}{\PYZpc{}d}\PY{l+s+s1}{.}\PY{l+s+s1}{\PYZpc{}}\PY{l+s+s1}{m}\PY{l+s+s1}{\PYZsq{}}\PY{p}{,} \PY{n}{utc}\PY{o}{=}\PY{k+kc}{False}\PY{p}{)}

    \PY{c+c1}{\PYZsh{} drop duplicate and keep newest videos}
    \PY{k}{return} \PY{n}{videos\PYZus{}clean}\PY{o}{.}\PY{n}{sort\PYZus{}values}\PY{p}{(}\PY{n}{by}\PY{o}{=}\PY{p}{[}\PY{l+s+s1}{\PYZsq{}}\PY{l+s+s1}{trending\PYZus{}date}\PY{l+s+s1}{\PYZsq{}}\PY{p}{]}\PY{p}{)}\PY{o}{.}\PY{n}{drop\PYZus{}duplicates}\PY{p}{(}\PY{n}{subset}\PY{o}{=}\PY{p}{[}\PY{l+s+s1}{\PYZsq{}}\PY{l+s+s1}{video\PYZus{}id}\PY{l+s+s1}{\PYZsq{}}\PY{p}{]}\PY{p}{,} \PY{n}{keep}\PY{o}{=}\PY{n}{keeped}\PY{p}{)}\PY{o}{.}\PY{n}{reset\PYZus{}index}\PY{p}{(}\PY{p}{)}

\PY{n}{FRvideos} \PY{o}{=} \PY{n}{clean\PYZus{}youtube\PYZus{}dataset}\PY{p}{(}\PY{l+s+s1}{\PYZsq{}}\PY{l+s+s1}{FRvideos.csv}\PY{l+s+s1}{\PYZsq{}}\PY{p}{,} \PY{l+s+s1}{\PYZsq{}}\PY{l+s+s1}{first}\PY{l+s+s1}{\PYZsq{}}\PY{p}{)}
\PY{n}{GBvideos} \PY{o}{=} \PY{n}{clean\PYZus{}youtube\PYZus{}dataset}\PY{p}{(}\PY{l+s+s1}{\PYZsq{}}\PY{l+s+s1}{GBvideos.csv}\PY{l+s+s1}{\PYZsq{}}\PY{p}{,} \PY{l+s+s1}{\PYZsq{}}\PY{l+s+s1}{first}\PY{l+s+s1}{\PYZsq{}}\PY{p}{)}
\PY{n}{DEvideos} \PY{o}{=} \PY{n}{clean\PYZus{}youtube\PYZus{}dataset}\PY{p}{(}\PY{l+s+s1}{\PYZsq{}}\PY{l+s+s1}{DEvideos.csv}\PY{l+s+s1}{\PYZsq{}}\PY{p}{,} \PY{l+s+s1}{\PYZsq{}}\PY{l+s+s1}{first}\PY{l+s+s1}{\PYZsq{}}\PY{p}{)}
\end{Verbatim}
\end{tcolorbox}

    \begin{tcolorbox}[breakable, size=fbox, boxrule=1pt, pad at break*=1mm,colback=cellbackground, colframe=cellborder]
\prompt{In}{incolor}{48}{\boxspacing}
\begin{Verbatim}[commandchars=\\\{\}]
\PY{k}{def} \PY{n+nf}{trending\PYZus{}time\PYZus{}mean\PYZus{}per\PYZus{}week}\PY{p}{(} \PY{n}{videos}\PY{p}{,} \PY{n}{title}\PY{p}{)}\PY{p}{:}
    \PY{n}{videos} \PY{o}{=} \PY{n}{videos}\PY{o}{.}\PY{n}{filter}\PY{p}{(}\PY{p}{[}\PY{l+s+s1}{\PYZsq{}}\PY{l+s+s1}{trending\PYZus{}date}\PY{l+s+s1}{\PYZsq{}}\PY{p}{,} \PY{l+s+s1}{\PYZsq{}}\PY{l+s+s1}{publish\PYZus{}time}\PY{l+s+s1}{\PYZsq{}}\PY{p}{]}\PY{p}{)}
    
    \PY{c+c1}{\PYZsh{} remove video before november 2017}
    \PY{n}{videos} \PY{o}{=} \PY{n}{videos}\PY{p}{[}\PY{n}{videos}\PY{p}{[}\PY{l+s+s1}{\PYZsq{}}\PY{l+s+s1}{publish\PYZus{}time}\PY{l+s+s1}{\PYZsq{}}\PY{p}{]} \PY{o}{\PYZgt{}} \PY{l+s+s1}{\PYZsq{}}\PY{l+s+s1}{2017\PYZhy{}11\PYZhy{}1}\PY{l+s+s1}{\PYZsq{}}\PY{p}{]}
    
    \PY{c+c1}{\PYZsh{} add column with day day of the week}
    \PY{n}{videos}\PY{p}{[}\PY{l+s+s1}{\PYZsq{}}\PY{l+s+s1}{day\PYZus{}of\PYZus{}the\PYZus{}week}\PY{l+s+s1}{\PYZsq{}}\PY{p}{]} \PY{o}{=} \PY{n}{videos}\PY{p}{[}\PY{l+s+s1}{\PYZsq{}}\PY{l+s+s1}{trending\PYZus{}date}\PY{l+s+s1}{\PYZsq{}}\PY{p}{]}\PY{o}{.}\PY{n}{dt}\PY{o}{.}\PY{n}{dayofweek}
    
    \PY{n}{videos}\PY{p}{[}\PY{l+s+s1}{\PYZsq{}}\PY{l+s+s1}{time\PYZus{}to\PYZus{}trended}\PY{l+s+s1}{\PYZsq{}}\PY{p}{]} \PY{o}{=} \PY{n}{videos}\PY{p}{[}\PY{l+s+s1}{\PYZsq{}}\PY{l+s+s1}{trending\PYZus{}date}\PY{l+s+s1}{\PYZsq{}}\PY{p}{]} \PY{o}{\PYZhy{}} \PY{n}{videos}\PY{p}{[}\PY{l+s+s1}{\PYZsq{}}\PY{l+s+s1}{publish\PYZus{}time}\PY{l+s+s1}{\PYZsq{}}\PY{p}{]}
    \PY{n}{videos}\PY{p}{[}\PY{l+s+s1}{\PYZsq{}}\PY{l+s+s1}{time\PYZus{}to\PYZus{}trended\PYZus{}seconds}\PY{l+s+s1}{\PYZsq{}}\PY{p}{]} \PY{o}{=} \PY{n}{videos}\PY{p}{[}\PY{l+s+s1}{\PYZsq{}}\PY{l+s+s1}{time\PYZus{}to\PYZus{}trended}\PY{l+s+s1}{\PYZsq{}}\PY{p}{]}\PY{o}{.}\PY{n}{dt}\PY{o}{.}\PY{n}{seconds} \PY{o}{+} \PY{n}{videos}\PY{p}{[}\PY{l+s+s1}{\PYZsq{}}\PY{l+s+s1}{time\PYZus{}to\PYZus{}trended}\PY{l+s+s1}{\PYZsq{}}\PY{p}{]}\PY{o}{.}\PY{n}{dt}\PY{o}{.}\PY{n}{days} \PY{o}{*} \PY{l+m+mi}{86400}
    \PY{n}{videos}\PY{p}{[}\PY{l+s+s1}{\PYZsq{}}\PY{l+s+s1}{time\PYZus{}to\PYZus{}trended\PYZus{}days}\PY{l+s+s1}{\PYZsq{}}\PY{p}{]} \PY{o}{=} \PY{n}{videos}\PY{p}{[}\PY{l+s+s1}{\PYZsq{}}\PY{l+s+s1}{time\PYZus{}to\PYZus{}trended\PYZus{}seconds}\PY{l+s+s1}{\PYZsq{}}\PY{p}{]} \PY{o}{/} \PY{p}{(}\PY{l+m+mi}{60} \PY{o}{*} \PY{l+m+mi}{60} \PY{o}{*} \PY{l+m+mi}{24}\PY{p}{)}

    \PY{n}{videos\PYZus{}per\PYZus{}day} \PY{o}{=} \PY{n}{videos}\PY{o}{.}\PY{n}{groupby}\PY{p}{(}\PY{l+s+s1}{\PYZsq{}}\PY{l+s+s1}{day\PYZus{}of\PYZus{}the\PYZus{}week}\PY{l+s+s1}{\PYZsq{}}\PY{p}{)}\PY{o}{.}\PY{n}{mean}\PY{p}{(}\PY{p}{)}
    \PY{n}{videos\PYZus{}per\PYZus{}day} \PY{o}{=} \PY{n}{videos\PYZus{}per\PYZus{}day}\PY{o}{.}\PY{n}{reset\PYZus{}index}\PY{p}{(}\PY{p}{)}
    
    \PY{c+c1}{\PYZsh{} convert number to day name}
    \PY{n}{days\PYZus{}of\PYZus{}week} \PY{o}{=} \PY{p}{\PYZob{}}\PY{l+m+mi}{0}\PY{p}{:} \PY{l+s+s1}{\PYZsq{}}\PY{l+s+s1}{Monday}\PY{l+s+s1}{\PYZsq{}}\PY{p}{,} \PY{l+m+mi}{1}\PY{p}{:} \PY{l+s+s1}{\PYZsq{}}\PY{l+s+s1}{Thuesday}\PY{l+s+s1}{\PYZsq{}}\PY{p}{,} \PY{l+m+mi}{2}\PY{p}{:} \PY{l+s+s1}{\PYZsq{}}\PY{l+s+s1}{Wenesday}\PY{l+s+s1}{\PYZsq{}}\PY{p}{,} \PY{l+m+mi}{3}\PY{p}{:} \PY{l+s+s1}{\PYZsq{}}\PY{l+s+s1}{Thursday}\PY{l+s+s1}{\PYZsq{}}\PY{p}{,} \PY{l+m+mi}{4}\PY{p}{:} \PY{l+s+s1}{\PYZsq{}}\PY{l+s+s1}{Friday}\PY{l+s+s1}{\PYZsq{}}\PY{p}{,} \PY{l+m+mi}{5}\PY{p}{:} \PY{l+s+s1}{\PYZsq{}}\PY{l+s+s1}{Saturday}\PY{l+s+s1}{\PYZsq{}}\PY{p}{,} \PY{l+m+mi}{6}\PY{p}{:} \PY{l+s+s1}{\PYZsq{}}\PY{l+s+s1}{Sunday}\PY{l+s+s1}{\PYZsq{}}\PY{p}{,}\PY{p}{\PYZcb{}}
    \PY{n}{videos\PYZus{}per\PYZus{}day}\PY{p}{[}\PY{l+s+s1}{\PYZsq{}}\PY{l+s+s1}{day\PYZus{}of\PYZus{}the\PYZus{}week}\PY{l+s+s1}{\PYZsq{}}\PY{p}{]} \PY{o}{=} \PY{n}{videos\PYZus{}per\PYZus{}day}\PY{p}{[}\PY{l+s+s1}{\PYZsq{}}\PY{l+s+s1}{day\PYZus{}of\PYZus{}the\PYZus{}week}\PY{l+s+s1}{\PYZsq{}}\PY{p}{]}\PY{o}{.}\PY{n}{map}\PY{p}{(}\PY{n}{days\PYZus{}of\PYZus{}week}\PY{p}{)}

    \PY{n}{videos\PYZus{}per\PYZus{}day}\PY{o}{.}\PY{n}{plot}\PY{p}{(}\PY{n}{kind}\PY{o}{=}\PY{l+s+s1}{\PYZsq{}}\PY{l+s+s1}{bar}\PY{l+s+s1}{\PYZsq{}}\PY{p}{,} \PY{n}{x}\PY{o}{=}\PY{l+s+s1}{\PYZsq{}}\PY{l+s+s1}{day\PYZus{}of\PYZus{}the\PYZus{}week}\PY{l+s+s1}{\PYZsq{}}\PY{p}{,} \PY{n}{y}\PY{o}{=}\PY{l+s+s1}{\PYZsq{}}\PY{l+s+s1}{time\PYZus{}to\PYZus{}trended\PYZus{}days}\PY{l+s+s1}{\PYZsq{}}\PY{p}{,} \PY{n}{color}\PY{o}{=}\PY{l+s+s1}{\PYZsq{}}\PY{l+s+s1}{red}\PY{l+s+s1}{\PYZsq{}}\PY{p}{,} \PY{n}{figsize}\PY{o}{=}\PY{p}{(}\PY{l+m+mi}{10}\PY{p}{,}\PY{l+m+mi}{5}\PY{p}{)}\PY{p}{)}
    \PY{n}{plt}\PY{o}{.}\PY{n}{title}\PY{p}{(}\PY{n}{title}\PY{p}{)}
    \PY{n}{plt}\PY{o}{.}\PY{n}{xticks}\PY{p}{(}\PY{n}{rotation}\PY{o}{=}\PY{l+m+mi}{0}\PY{p}{)}
    \PY{n}{plt}\PY{o}{.}\PY{n}{xlabel}\PY{p}{(}\PY{l+s+s2}{\PYZdq{}}\PY{l+s+s2}{Jour de la semaine}\PY{l+s+s2}{\PYZdq{}}\PY{p}{)}
    \PY{n}{plt}\PY{o}{.}\PY{n}{ylabel}\PY{p}{(}\PY{l+s+s2}{\PYZdq{}}\PY{l+s+s2}{Temps de publication moyen en jour}\PY{l+s+s2}{\PYZdq{}}\PY{p}{)}

    \PY{n}{plt}\PY{o}{.}\PY{n}{show}\PY{p}{(}\PY{p}{)}
\end{Verbatim}
\end{tcolorbox}

    Le premier choix fait a du etre de retirer les videos les plus vieilles
j'ai donc choisi de prendre uniquement les vidéos publiées après
novembre 2017, soit 2 semaine avant la premier date de tendance dans
notre dataset.

\begin{Shaded}
\begin{Highlighting}[]
\CommentTok{\# remove video before november 2017}
\NormalTok{videos }\OperatorTok{=}\NormalTok{ videos[videos[}\StringTok{\textquotesingle{}publish\_time\textquotesingle{}}\NormalTok{] }\OperatorTok{\textgreater{}} \StringTok{\textquotesingle{}2017{-}11{-}1\textquotesingle{}}\NormalTok{]}
\end{Highlighting}
\end{Shaded}

Cette opération a par exemple beaucoup corriger l'écart type dans le
dataset d'Angleterre. Nous nous sommes rendu compte que beaucoup de
video en tendance pendant cette période était ancienne .

    \begin{tcolorbox}[breakable, size=fbox, boxrule=1pt, pad at break*=1mm,colback=cellbackground, colframe=cellborder]
\prompt{In}{incolor}{49}{\boxspacing}
\begin{Verbatim}[commandchars=\\\{\}]
\PY{n}{trending\PYZus{}time\PYZus{}mean\PYZus{}per\PYZus{}week}\PY{p}{(}\PY{n}{FRvideos}\PY{p}{,} \PY{l+s+s2}{\PYZdq{}}\PY{l+s+s2}{Moyenne du temps de publication d}\PY{l+s+s2}{\PYZsq{}}\PY{l+s+s2}{une video en tendance par jour de la semaine en France.}\PY{l+s+s2}{\PYZdq{}}\PY{p}{)}
\end{Verbatim}
\end{tcolorbox}

    \begin{center}
    \adjustimage{max size={0.9\linewidth}{0.9\paperheight}}{projet_files/projet_26_0.png}
    \end{center}
    { \hspace*{\fill} \\}
    
    \begin{tcolorbox}[breakable, size=fbox, boxrule=1pt, pad at break*=1mm,colback=cellbackground, colframe=cellborder]
\prompt{In}{incolor}{50}{\boxspacing}
\begin{Verbatim}[commandchars=\\\{\}]
\PY{n}{trending\PYZus{}time\PYZus{}mean\PYZus{}per\PYZus{}week}\PY{p}{(}\PY{n}{DEvideos}\PY{p}{,} \PY{l+s+s2}{\PYZdq{}}\PY{l+s+s2}{Moyenne du temps de publication d}\PY{l+s+s2}{\PYZsq{}}\PY{l+s+s2}{une video en tendance par jour de la semaine en Allemagne.}\PY{l+s+s2}{\PYZdq{}}\PY{p}{)}
\end{Verbatim}
\end{tcolorbox}

    \begin{center}
    \adjustimage{max size={0.9\linewidth}{0.9\paperheight}}{projet_files/projet_27_0.png}
    \end{center}
    { \hspace*{\fill} \\}
    
    \begin{tcolorbox}[breakable, size=fbox, boxrule=1pt, pad at break*=1mm,colback=cellbackground, colframe=cellborder]
\prompt{In}{incolor}{51}{\boxspacing}
\begin{Verbatim}[commandchars=\\\{\}]
\PY{n}{trending\PYZus{}time\PYZus{}mean\PYZus{}per\PYZus{}week}\PY{p}{(}\PY{n}{GBvideos}\PY{p}{,} \PY{l+s+s2}{\PYZdq{}}\PY{l+s+s2}{Moyenne du temps de publication d}\PY{l+s+s2}{\PYZsq{}}\PY{l+s+s2}{une video en tendance par jour de la semaine en Angleterre.}\PY{l+s+s2}{\PYZdq{}}\PY{p}{)}
\end{Verbatim}
\end{tcolorbox}

    \begin{center}
    \adjustimage{max size={0.9\linewidth}{0.9\paperheight}}{projet_files/projet_28_0.png}
    \end{center}
    { \hspace*{\fill} \\}
    
    \hypertarget{analyse}{%
\subsubsection{3.2.2 Analyse}\label{analyse}}

On remaque peu de difference entre le dataset d'Allemagne est de la
France la tendance est d'environ 12h, c'est-à-dire que les video sont en
moyenne posté 12h a l'avance. Les données en angleterre sont beaucoup
plus intéréssante. Tout d'abort on remarque que la moyenne est beaucoup
plus proche de 3 jours. On note aussi que le mercredi, le jeudi et le
vendredi les moyennes sont nettement inférieur au autre jour, environ 1
jour et 12 heures. On peut donc conclure que les videos poster le lundi
est le mardi sont plus propice à entrer en tendance.

    \hypertarget{etude-sur-la-signification-des-tags}{%
\subsection{3.3 Etude sur la signification des
tags}\label{etude-sur-la-signification-des-tags}}

\hypertarget{objectif-de-lanalyse}{%
\subsubsection{3.3.1 Objectif de l'analyse}\label{objectif-de-lanalyse}}

L'objectif de cette analyse est de comprendre comment évolue les tags
des vidéos celon les dates de publication de celle ci, on pourra ainsi
mettre en évidence l'influence de certaine période de l'année sur les
tags ou encore d'événement marquant.

    \hypertarget{nettoyage-des-donnuxe9es}{%
\subsubsection{3.3.2 Nettoyage des
données}\label{nettoyage-des-donnuxe9es}}

Dans un premier temps on nettois les données du Dataframe sélectionné en
retirant les colonnes non utile à la comprehension du dataframe, on met
ensuite les dates du dataframe a format datetime de python pour
faciliter le traitement des données. Ensuite on supprime les doublons du
dataframe.

    \begin{tcolorbox}[breakable, size=fbox, boxrule=1pt, pad at break*=1mm,colback=cellbackground, colframe=cellborder]
\prompt{In}{incolor}{52}{\boxspacing}
\begin{Verbatim}[commandchars=\\\{\}]
\PY{n}{videos\PYZus{}cleanfr} \PY{o}{=} \PY{n}{clean\PYZus{}youtube\PYZus{}dataset}\PY{p}{(}\PY{l+s+s1}{\PYZsq{}}\PY{l+s+s1}{FRvideos.csv}\PY{l+s+s1}{\PYZsq{}}\PY{p}{,} \PY{l+s+s1}{\PYZsq{}}\PY{l+s+s1}{last}\PY{l+s+s1}{\PYZsq{}}\PY{p}{)}
\PY{n}{videos\PYZus{}cleande} \PY{o}{=} \PY{n}{clean\PYZus{}youtube\PYZus{}dataset}\PY{p}{(}\PY{l+s+s1}{\PYZsq{}}\PY{l+s+s1}{GBvideos.csv}\PY{l+s+s1}{\PYZsq{}}\PY{p}{,} \PY{l+s+s1}{\PYZsq{}}\PY{l+s+s1}{last}\PY{l+s+s1}{\PYZsq{}}\PY{p}{)}
\PY{n}{videos\PYZus{}cleangb} \PY{o}{=} \PY{n}{clean\PYZus{}youtube\PYZus{}dataset}\PY{p}{(}\PY{l+s+s1}{\PYZsq{}}\PY{l+s+s1}{DEvideos.csv}\PY{l+s+s1}{\PYZsq{}}\PY{p}{,} \PY{l+s+s1}{\PYZsq{}}\PY{l+s+s1}{last}\PY{l+s+s1}{\PYZsq{}}\PY{p}{)}
\end{Verbatim}
\end{tcolorbox}

    \emph{La fonction explode permet de convertir une colonne de liste en
une colonne avec toute les elements de toute les listes}

    \begin{tcolorbox}[breakable, size=fbox, boxrule=1pt, pad at break*=1mm,colback=cellbackground, colframe=cellborder]
\prompt{In}{incolor}{53}{\boxspacing}
\begin{Verbatim}[commandchars=\\\{\}]
\PY{k}{def} \PY{n+nf}{explode}\PY{p}{(}\PY{n}{df}\PY{p}{,} \PY{n}{lst\PYZus{}cols}\PY{p}{,} \PY{n}{fill\PYZus{}value}\PY{o}{=}\PY{l+s+s1}{\PYZsq{}}\PY{l+s+s1}{\PYZsq{}}\PY{p}{,} \PY{n}{preserve\PYZus{}index}\PY{o}{=}\PY{k+kc}{False}\PY{p}{)}\PY{p}{:}
    \PY{c+c1}{\PYZsh{} make sure `lst\PYZus{}cols` is list\PYZhy{}alike}
    \PY{k}{if} \PY{p}{(}\PY{n}{lst\PYZus{}cols} \PY{o+ow}{is} \PY{o+ow}{not} \PY{k+kc}{None}
        \PY{o+ow}{and} \PY{n+nb}{len}\PY{p}{(}\PY{n}{lst\PYZus{}cols}\PY{p}{)} \PY{o}{\PYZgt{}} \PY{l+m+mi}{0}
        \PY{o+ow}{and} \PY{o+ow}{not} \PY{n+nb}{isinstance}\PY{p}{(}\PY{n}{lst\PYZus{}cols}\PY{p}{,} \PY{p}{(}\PY{n+nb}{list}\PY{p}{,} \PY{n+nb}{tuple}\PY{p}{,} \PY{n}{np}\PY{o}{.}\PY{n}{ndarray}\PY{p}{,} \PY{n}{pd}\PY{o}{.}\PY{n}{Series}\PY{p}{)}\PY{p}{)}\PY{p}{)}\PY{p}{:}
        \PY{n}{lst\PYZus{}cols} \PY{o}{=} \PY{p}{[}\PY{n}{lst\PYZus{}cols}\PY{p}{]}
    \PY{c+c1}{\PYZsh{} all columns except `lst\PYZus{}cols`}
    \PY{n}{idx\PYZus{}cols} \PY{o}{=} \PY{n}{df}\PY{o}{.}\PY{n}{columns}\PY{o}{.}\PY{n}{difference}\PY{p}{(}\PY{n}{lst\PYZus{}cols}\PY{p}{)}
    \PY{c+c1}{\PYZsh{} calculate lengths of lists}
    \PY{n}{lens} \PY{o}{=} \PY{n}{df}\PY{p}{[}\PY{n}{lst\PYZus{}cols}\PY{p}{[}\PY{l+m+mi}{0}\PY{p}{]}\PY{p}{]}\PY{o}{.}\PY{n}{str}\PY{o}{.}\PY{n}{len}\PY{p}{(}\PY{p}{)}
    \PY{c+c1}{\PYZsh{}cv preserve original index values    }
    \PY{n}{idx} \PY{o}{=} \PY{n}{np}\PY{o}{.}\PY{n}{repeat}\PY{p}{(}\PY{n}{df}\PY{o}{.}\PY{n}{index}\PY{o}{.}\PY{n}{values}\PY{p}{,} \PY{n}{lens}\PY{p}{)}
    \PY{c+c1}{\PYZsh{} create \PYZdq{}exploded\PYZdq{} DF}
    \PY{n}{res} \PY{o}{=} \PY{p}{(}\PY{n}{pd}\PY{o}{.}\PY{n}{DataFrame}\PY{p}{(}\PY{p}{\PYZob{}}
                \PY{n}{col}\PY{p}{:}\PY{n}{np}\PY{o}{.}\PY{n}{repeat}\PY{p}{(}\PY{n}{df}\PY{p}{[}\PY{n}{col}\PY{p}{]}\PY{o}{.}\PY{n}{values}\PY{p}{,} \PY{n}{lens}\PY{p}{)}
                \PY{k}{for} \PY{n}{col} \PY{o+ow}{in} \PY{n}{idx\PYZus{}cols}\PY{p}{\PYZcb{}}\PY{p}{,}
                \PY{n}{index}\PY{o}{=}\PY{n}{idx}\PY{p}{)}
             \PY{o}{.}\PY{n}{assign}\PY{p}{(}\PY{o}{*}\PY{o}{*}\PY{p}{\PYZob{}}\PY{n}{col}\PY{p}{:}\PY{n}{np}\PY{o}{.}\PY{n}{concatenate}\PY{p}{(}\PY{n}{df}\PY{o}{.}\PY{n}{loc}\PY{p}{[}\PY{n}{lens}\PY{o}{\PYZgt{}}\PY{l+m+mi}{0}\PY{p}{,} \PY{n}{col}\PY{p}{]}\PY{o}{.}\PY{n}{values}\PY{p}{)}
                            \PY{k}{for} \PY{n}{col} \PY{o+ow}{in} \PY{n}{lst\PYZus{}cols}\PY{p}{\PYZcb{}}\PY{p}{)}\PY{p}{)}
    \PY{c+c1}{\PYZsh{} append those rows that have empty lists}
    \PY{k}{if} \PY{p}{(}\PY{n}{lens} \PY{o}{==} \PY{l+m+mi}{0}\PY{p}{)}\PY{o}{.}\PY{n}{any}\PY{p}{(}\PY{p}{)}\PY{p}{:}
        \PY{c+c1}{\PYZsh{} at least one list in cells is empty}
        \PY{n}{res} \PY{o}{=} \PY{p}{(}\PY{n}{res}\PY{o}{.}\PY{n}{append}\PY{p}{(}\PY{n}{df}\PY{o}{.}\PY{n}{loc}\PY{p}{[}\PY{n}{lens}\PY{o}{==}\PY{l+m+mi}{0}\PY{p}{,} \PY{n}{idx\PYZus{}cols}\PY{p}{]}\PY{p}{,} \PY{n}{sort}\PY{o}{=}\PY{k+kc}{False}\PY{p}{)}
                  \PY{o}{.}\PY{n}{fillna}\PY{p}{(}\PY{n}{fill\PYZus{}value}\PY{p}{)}\PY{p}{)}
    \PY{c+c1}{\PYZsh{} revert the original index order}
    \PY{n}{res} \PY{o}{=} \PY{n}{res}\PY{o}{.}\PY{n}{sort\PYZus{}index}\PY{p}{(}\PY{p}{)}
    \PY{c+c1}{\PYZsh{} reset index if requested}
    \PY{k}{if} \PY{o+ow}{not} \PY{n}{preserve\PYZus{}index}\PY{p}{:}        
        \PY{n}{res} \PY{o}{=} \PY{n}{res}\PY{o}{.}\PY{n}{reset\PYZus{}index}\PY{p}{(}\PY{n}{drop}\PY{o}{=}\PY{k+kc}{True}\PY{p}{)}
    \PY{k}{return} \PY{n}{res}
\end{Verbatim}
\end{tcolorbox}

    \hypertarget{analyse-des-donnuxe9es}{%
\subsubsection{3.3.3 Analyse des données}\label{analyse-des-donnuxe9es}}

On peut tout d'abord visionner les tags les plus populaires de chaque
dataframe sans les visionner par mois pour vérifier si les tags les plus
populaires de l'année se retrouve ensuite dans les différent mois.

    \begin{tcolorbox}[breakable, size=fbox, boxrule=1pt, pad at break*=1mm,colback=cellbackground, colframe=cellborder]
\prompt{In}{incolor}{54}{\boxspacing}
\begin{Verbatim}[commandchars=\\\{\}]
\PY{n}{tagsfr} \PY{o}{=} \PY{n}{videos\PYZus{}cleanfr}\PY{o}{.}\PY{n}{filter}\PY{p}{(}\PY{p}{[}\PY{l+s+s1}{\PYZsq{}}\PY{l+s+s1}{tags}\PY{l+s+s1}{\PYZsq{}}\PY{p}{]}\PY{p}{)}
\PY{n}{tagsde} \PY{o}{=} \PY{n}{videos\PYZus{}cleande}\PY{o}{.}\PY{n}{filter}\PY{p}{(}\PY{p}{[}\PY{l+s+s1}{\PYZsq{}}\PY{l+s+s1}{tags}\PY{l+s+s1}{\PYZsq{}}\PY{p}{]}\PY{p}{)}
\PY{n}{tagsgb} \PY{o}{=} \PY{n}{videos\PYZus{}cleangb}\PY{o}{.}\PY{n}{filter}\PY{p}{(}\PY{p}{[}\PY{l+s+s1}{\PYZsq{}}\PY{l+s+s1}{tags}\PY{l+s+s1}{\PYZsq{}}\PY{p}{]}\PY{p}{)}

\PY{n}{tagsfr}\PY{p}{[}\PY{l+s+s1}{\PYZsq{}}\PY{l+s+s1}{tags}\PY{l+s+s1}{\PYZsq{}}\PY{p}{]} \PY{o}{=} \PY{n}{tagsfr}\PY{p}{[}\PY{l+s+s1}{\PYZsq{}}\PY{l+s+s1}{tags}\PY{l+s+s1}{\PYZsq{}}\PY{p}{]}\PY{o}{.}\PY{n}{str}\PY{o}{.}\PY{n}{replace}\PY{p}{(}\PY{l+s+s1}{\PYZsq{}}\PY{l+s+s1}{\PYZdq{}}\PY{l+s+s1}{\PYZsq{}}\PY{p}{,}\PY{l+s+s1}{\PYZsq{}}\PY{l+s+s1}{\PYZsq{}}\PY{p}{)}
\PY{n}{tagsfr}\PY{p}{[}\PY{l+s+s1}{\PYZsq{}}\PY{l+s+s1}{tags}\PY{l+s+s1}{\PYZsq{}}\PY{p}{]} \PY{o}{=} \PY{n}{tagsfr}\PY{p}{[}\PY{l+s+s1}{\PYZsq{}}\PY{l+s+s1}{tags}\PY{l+s+s1}{\PYZsq{}}\PY{p}{]}\PY{o}{.}\PY{n}{str}\PY{o}{.}\PY{n}{split}\PY{p}{(}\PY{l+s+s1}{\PYZsq{}}\PY{l+s+s1}{|}\PY{l+s+s1}{\PYZsq{}}\PY{p}{)}
\PY{n}{tagsfr} \PY{o}{=} \PY{n}{explode}\PY{p}{(}\PY{n}{tagsfr}\PY{p}{,} \PY{p}{[}\PY{l+s+s1}{\PYZsq{}}\PY{l+s+s1}{tags}\PY{l+s+s1}{\PYZsq{}}\PY{p}{]}\PY{p}{,} \PY{n}{fill\PYZus{}value}\PY{o}{=}\PY{l+s+s1}{\PYZsq{}}\PY{l+s+s1}{\PYZsq{}}\PY{p}{)}
\PY{n}{tags\PYZus{}most\PYZus{}usedfr} \PY{o}{=} \PY{n}{tagsfr}\PY{p}{[}\PY{l+s+s1}{\PYZsq{}}\PY{l+s+s1}{tags}\PY{l+s+s1}{\PYZsq{}}\PY{p}{]}\PY{o}{.}\PY{n}{value\PYZus{}counts}\PY{p}{(}\PY{p}{)}\PY{o}{.}\PY{n}{head}\PY{p}{(}\PY{l+m+mi}{25}\PY{p}{)}\PY{o}{.}\PY{n}{sort\PYZus{}values}\PY{p}{(}\PY{p}{)}
\PY{n}{tags\PYZus{}most\PYZus{}usedfr}\PY{o}{.}\PY{n}{plot}\PY{p}{(}\PY{n}{kind}\PY{o}{=}\PY{l+s+s1}{\PYZsq{}}\PY{l+s+s1}{barh}\PY{l+s+s1}{\PYZsq{}}\PY{p}{,}
                    \PY{n}{figsize}\PY{o}{=}\PY{p}{(}\PY{l+m+mi}{10}\PY{p}{,}\PY{l+m+mi}{10}\PY{p}{)}\PY{p}{)}
\PY{n}{plt}\PY{o}{.}\PY{n}{title}\PY{p}{(}\PY{l+s+s1}{\PYZsq{}}\PY{l+s+s1}{Tags les plus populaire en France}\PY{l+s+s1}{\PYZsq{}}\PY{p}{)}
\PY{n}{plt}\PY{o}{.}\PY{n}{xlabel}\PY{p}{(}\PY{l+s+s2}{\PYZdq{}}\PY{l+s+s2}{nombre de tags}\PY{l+s+s2}{\PYZdq{}}\PY{p}{)}
\PY{n}{plt}\PY{o}{.}\PY{n}{show}\PY{p}{(}\PY{p}{)}

\PY{n}{tagsde}\PY{p}{[}\PY{l+s+s1}{\PYZsq{}}\PY{l+s+s1}{tags}\PY{l+s+s1}{\PYZsq{}}\PY{p}{]} \PY{o}{=} \PY{n}{tagsde}\PY{p}{[}\PY{l+s+s1}{\PYZsq{}}\PY{l+s+s1}{tags}\PY{l+s+s1}{\PYZsq{}}\PY{p}{]}\PY{o}{.}\PY{n}{str}\PY{o}{.}\PY{n}{replace}\PY{p}{(}\PY{l+s+s1}{\PYZsq{}}\PY{l+s+s1}{\PYZdq{}}\PY{l+s+s1}{\PYZsq{}}\PY{p}{,}\PY{l+s+s1}{\PYZsq{}}\PY{l+s+s1}{\PYZsq{}}\PY{p}{)}
\PY{n}{tagsde}\PY{p}{[}\PY{l+s+s1}{\PYZsq{}}\PY{l+s+s1}{tags}\PY{l+s+s1}{\PYZsq{}}\PY{p}{]} \PY{o}{=} \PY{n}{tagsde}\PY{p}{[}\PY{l+s+s1}{\PYZsq{}}\PY{l+s+s1}{tags}\PY{l+s+s1}{\PYZsq{}}\PY{p}{]}\PY{o}{.}\PY{n}{str}\PY{o}{.}\PY{n}{split}\PY{p}{(}\PY{l+s+s1}{\PYZsq{}}\PY{l+s+s1}{|}\PY{l+s+s1}{\PYZsq{}}\PY{p}{)}
\PY{n}{tagsde} \PY{o}{=} \PY{n}{explode}\PY{p}{(}\PY{n}{tagsde}\PY{p}{,} \PY{p}{[}\PY{l+s+s1}{\PYZsq{}}\PY{l+s+s1}{tags}\PY{l+s+s1}{\PYZsq{}}\PY{p}{]}\PY{p}{,} \PY{n}{fill\PYZus{}value}\PY{o}{=}\PY{l+s+s1}{\PYZsq{}}\PY{l+s+s1}{\PYZsq{}}\PY{p}{)}
\PY{n}{tags\PYZus{}most\PYZus{}usedde} \PY{o}{=} \PY{n}{tagsde}\PY{p}{[}\PY{l+s+s1}{\PYZsq{}}\PY{l+s+s1}{tags}\PY{l+s+s1}{\PYZsq{}}\PY{p}{]}\PY{o}{.}\PY{n}{value\PYZus{}counts}\PY{p}{(}\PY{p}{)}\PY{o}{.}\PY{n}{head}\PY{p}{(}\PY{l+m+mi}{25}\PY{p}{)}\PY{o}{.}\PY{n}{sort\PYZus{}values}\PY{p}{(}\PY{p}{)}
\PY{n}{tags\PYZus{}most\PYZus{}usedde}\PY{o}{.}\PY{n}{plot}\PY{p}{(}\PY{n}{kind}\PY{o}{=}\PY{l+s+s1}{\PYZsq{}}\PY{l+s+s1}{barh}\PY{l+s+s1}{\PYZsq{}}\PY{p}{,}
                    \PY{n}{figsize}\PY{o}{=}\PY{p}{(}\PY{l+m+mi}{10}\PY{p}{,}\PY{l+m+mi}{10}\PY{p}{)}\PY{p}{)}
\PY{n}{plt}\PY{o}{.}\PY{n}{title}\PY{p}{(}\PY{l+s+s1}{\PYZsq{}}\PY{l+s+s1}{Tags les plus populaire en Allemagne}\PY{l+s+s1}{\PYZsq{}}\PY{p}{)}
\PY{n}{plt}\PY{o}{.}\PY{n}{xlabel}\PY{p}{(}\PY{l+s+s2}{\PYZdq{}}\PY{l+s+s2}{nombre de tags}\PY{l+s+s2}{\PYZdq{}}\PY{p}{)}
\PY{n}{plt}\PY{o}{.}\PY{n}{show}\PY{p}{(}\PY{p}{)}

\PY{n}{tagsgb}\PY{p}{[}\PY{l+s+s1}{\PYZsq{}}\PY{l+s+s1}{tags}\PY{l+s+s1}{\PYZsq{}}\PY{p}{]} \PY{o}{=} \PY{n}{tagsgb}\PY{p}{[}\PY{l+s+s1}{\PYZsq{}}\PY{l+s+s1}{tags}\PY{l+s+s1}{\PYZsq{}}\PY{p}{]}\PY{o}{.}\PY{n}{str}\PY{o}{.}\PY{n}{replace}\PY{p}{(}\PY{l+s+s1}{\PYZsq{}}\PY{l+s+s1}{\PYZdq{}}\PY{l+s+s1}{\PYZsq{}}\PY{p}{,}\PY{l+s+s1}{\PYZsq{}}\PY{l+s+s1}{\PYZsq{}}\PY{p}{)}
\PY{n}{tagsgb}\PY{p}{[}\PY{l+s+s1}{\PYZsq{}}\PY{l+s+s1}{tags}\PY{l+s+s1}{\PYZsq{}}\PY{p}{]} \PY{o}{=} \PY{n}{tagsgb}\PY{p}{[}\PY{l+s+s1}{\PYZsq{}}\PY{l+s+s1}{tags}\PY{l+s+s1}{\PYZsq{}}\PY{p}{]}\PY{o}{.}\PY{n}{str}\PY{o}{.}\PY{n}{split}\PY{p}{(}\PY{l+s+s1}{\PYZsq{}}\PY{l+s+s1}{|}\PY{l+s+s1}{\PYZsq{}}\PY{p}{)}
\PY{n}{tagsgb} \PY{o}{=} \PY{n}{explode}\PY{p}{(}\PY{n}{tagsgb}\PY{p}{,} \PY{p}{[}\PY{l+s+s1}{\PYZsq{}}\PY{l+s+s1}{tags}\PY{l+s+s1}{\PYZsq{}}\PY{p}{]}\PY{p}{,} \PY{n}{fill\PYZus{}value}\PY{o}{=}\PY{l+s+s1}{\PYZsq{}}\PY{l+s+s1}{\PYZsq{}}\PY{p}{)}
\PY{n}{tags\PYZus{}most\PYZus{}usedgb} \PY{o}{=} \PY{n}{tagsgb}\PY{p}{[}\PY{l+s+s1}{\PYZsq{}}\PY{l+s+s1}{tags}\PY{l+s+s1}{\PYZsq{}}\PY{p}{]}\PY{o}{.}\PY{n}{value\PYZus{}counts}\PY{p}{(}\PY{p}{)}\PY{o}{.}\PY{n}{head}\PY{p}{(}\PY{l+m+mi}{25}\PY{p}{)}\PY{o}{.}\PY{n}{sort\PYZus{}values}\PY{p}{(}\PY{p}{)}
\PY{n}{tags\PYZus{}most\PYZus{}usedgb}\PY{o}{.}\PY{n}{plot}\PY{p}{(}\PY{n}{kind}\PY{o}{=}\PY{l+s+s1}{\PYZsq{}}\PY{l+s+s1}{barh}\PY{l+s+s1}{\PYZsq{}}\PY{p}{,}
                    \PY{n}{figsize}\PY{o}{=}\PY{p}{(}\PY{l+m+mi}{10}\PY{p}{,}\PY{l+m+mi}{10}\PY{p}{)}\PY{p}{)}
\PY{n}{plt}\PY{o}{.}\PY{n}{title}\PY{p}{(}\PY{l+s+s1}{\PYZsq{}}\PY{l+s+s1}{Tags les plus populaire en Angleterre}\PY{l+s+s1}{\PYZsq{}}\PY{p}{)}
\PY{n}{plt}\PY{o}{.}\PY{n}{xlabel}\PY{p}{(}\PY{l+s+s2}{\PYZdq{}}\PY{l+s+s2}{nombre de tags}\PY{l+s+s2}{\PYZdq{}}\PY{p}{)}
\PY{n}{plt}\PY{o}{.}\PY{n}{show}\PY{p}{(}\PY{p}{)}
\end{Verbatim}
\end{tcolorbox}

    \begin{center}
    \adjustimage{max size={0.9\linewidth}{0.9\paperheight}}{projet_files/projet_36_0.png}
    \end{center}
    { \hspace*{\fill} \\}
    
    \begin{center}
    \adjustimage{max size={0.9\linewidth}{0.9\paperheight}}{projet_files/projet_36_1.png}
    \end{center}
    { \hspace*{\fill} \\}
    
    \begin{center}
    \adjustimage{max size={0.9\linewidth}{0.9\paperheight}}{projet_files/projet_36_2.png}
    \end{center}
    { \hspace*{\fill} \\}
    
    Dans la cellule précedente on peut remarquer que le ``tag'' le plus
utilisé dans les vidéos Allemande et Francaise en tendance est le tag
{[}none{]} en réalité cela signifie que la plupart du temps ces vidéos
ne comporte pas de tag, mais est-ce le cas si on visionne l'utilisation
des tags par mois? C'est ce que l'on va visionner maintenant :

    \begin{tcolorbox}[breakable, size=fbox, boxrule=1pt, pad at break*=1mm,colback=cellbackground, colframe=cellborder]
\prompt{In}{incolor}{55}{\boxspacing}
\begin{Verbatim}[commandchars=\\\{\}]
\PY{k}{def} \PY{n+nf}{display\PYZus{}tag\PYZus{}per\PYZus{}country}\PY{p}{(}\PY{n}{videos\PYZus{}clean}\PY{p}{)}\PY{p}{:}
    \PY{n}{tags\PYZus{}days} \PY{o}{=} \PY{n}{videos\PYZus{}clean}\PY{o}{.}\PY{n}{filter}\PY{p}{(}\PY{p}{[}\PY{l+s+s1}{\PYZsq{}}\PY{l+s+s1}{tags}\PY{l+s+s1}{\PYZsq{}}\PY{p}{,} \PY{l+s+s1}{\PYZsq{}}\PY{l+s+s1}{trending\PYZus{}date}\PY{l+s+s1}{\PYZsq{}}\PY{p}{]}\PY{p}{)}
    \PY{n}{tags\PYZus{}days}\PY{p}{[}\PY{l+s+s1}{\PYZsq{}}\PY{l+s+s1}{trending\PYZus{}date}\PY{l+s+s1}{\PYZsq{}}\PY{p}{]} \PY{o}{=} \PY{n}{tags\PYZus{}days}\PY{p}{[}\PY{l+s+s1}{\PYZsq{}}\PY{l+s+s1}{trending\PYZus{}date}\PY{l+s+s1}{\PYZsq{}}\PY{p}{]}\PY{o}{.}\PY{n}{dt}\PY{o}{.}\PY{n}{to\PYZus{}period}\PY{p}{(}\PY{l+s+s1}{\PYZsq{}}\PY{l+s+s1}{M}\PY{l+s+s1}{\PYZsq{}}\PY{p}{)}

    \PY{n}{tags\PYZus{}days}\PY{p}{[}\PY{l+s+s1}{\PYZsq{}}\PY{l+s+s1}{tags}\PY{l+s+s1}{\PYZsq{}}\PY{p}{]} \PY{o}{=} \PY{n}{tags\PYZus{}days}\PY{p}{[}\PY{l+s+s1}{\PYZsq{}}\PY{l+s+s1}{tags}\PY{l+s+s1}{\PYZsq{}}\PY{p}{]}\PY{o}{.}\PY{n}{str}\PY{o}{.}\PY{n}{lower}\PY{p}{(}\PY{p}{)}
    \PY{n}{tags\PYZus{}days}\PY{p}{[}\PY{l+s+s1}{\PYZsq{}}\PY{l+s+s1}{tags}\PY{l+s+s1}{\PYZsq{}}\PY{p}{]} \PY{o}{=} \PY{n}{tags\PYZus{}days}\PY{p}{[}\PY{l+s+s1}{\PYZsq{}}\PY{l+s+s1}{tags}\PY{l+s+s1}{\PYZsq{}}\PY{p}{]}\PY{o}{.}\PY{n}{str}\PY{o}{.}\PY{n}{replace}\PY{p}{(}\PY{l+s+s1}{\PYZsq{}}\PY{l+s+s1}{\PYZdq{}}\PY{l+s+s1}{\PYZsq{}}\PY{p}{,}\PY{l+s+s1}{\PYZsq{}}\PY{l+s+s1}{\PYZsq{}}\PY{p}{)}
    \PY{n}{tags\PYZus{}days}\PY{p}{[}\PY{l+s+s1}{\PYZsq{}}\PY{l+s+s1}{tags}\PY{l+s+s1}{\PYZsq{}}\PY{p}{]} \PY{o}{=} \PY{n}{tags\PYZus{}days}\PY{p}{[}\PY{l+s+s1}{\PYZsq{}}\PY{l+s+s1}{tags}\PY{l+s+s1}{\PYZsq{}}\PY{p}{]}\PY{o}{.}\PY{n}{str}\PY{o}{.}\PY{n}{split}\PY{p}{(}\PY{l+s+s1}{\PYZsq{}}\PY{l+s+s1}{|}\PY{l+s+s1}{\PYZsq{}}\PY{p}{)}
    \PY{n}{tags\PYZus{}days} \PY{o}{=} \PY{n}{explode}\PY{p}{(}\PY{n}{tags\PYZus{}days}\PY{p}{,} \PY{p}{[}\PY{l+s+s1}{\PYZsq{}}\PY{l+s+s1}{tags}\PY{l+s+s1}{\PYZsq{}}\PY{p}{]}\PY{p}{,} \PY{n}{fill\PYZus{}value}\PY{o}{=}\PY{l+s+s1}{\PYZsq{}}\PY{l+s+s1}{\PYZsq{}}\PY{p}{)}

    \PY{n}{tags\PYZus{}month\PYZus{}group} \PY{o}{=} \PY{n}{tags\PYZus{}days}\PY{o}{.}\PY{n}{groupby}\PY{p}{(}\PY{n}{by}\PY{o}{=}\PY{l+s+s1}{\PYZsq{}}\PY{l+s+s1}{trending\PYZus{}date}\PY{l+s+s1}{\PYZsq{}}\PY{p}{)}

    \PY{n}{figure}\PY{p}{,} \PY{n}{axes} \PY{o}{=} \PY{n}{plt}\PY{o}{.}\PY{n}{subplots}\PY{p}{(}\PY{l+m+mi}{4}\PY{p}{,} \PY{l+m+mi}{2}\PY{p}{)}
    \PY{k}{for} \PY{n}{index}\PY{p}{,} \PY{n}{month} \PY{o+ow}{in} \PY{n+nb}{enumerate}\PY{p}{(}\PY{n}{tags\PYZus{}month\PYZus{}group}\PY{o}{.}\PY{n}{groups}\PY{o}{.}\PY{n}{keys}\PY{p}{(}\PY{p}{)}\PY{p}{)}\PY{p}{:}
        \PY{n}{tags\PYZus{}month} \PY{o}{=} \PY{n}{tags\PYZus{}month\PYZus{}group}\PY{o}{.}\PY{n}{get\PYZus{}group}\PY{p}{(}\PY{n}{month}\PY{p}{)}
        \PY{n}{tags\PYZus{}frequency} \PY{o}{=} \PY{n}{tags\PYZus{}month}\PY{p}{[}\PY{l+s+s1}{\PYZsq{}}\PY{l+s+s1}{tags}\PY{l+s+s1}{\PYZsq{}}\PY{p}{]}\PY{o}{.}\PY{n}{value\PYZus{}counts}\PY{p}{(}\PY{p}{)}\PY{o}{.}\PY{n}{head}\PY{p}{(}\PY{l+m+mi}{20}\PY{p}{)}

        \PY{c+c1}{\PYZsh{}tags\PYZus{}frequency = tags\PYZus{}frequency.drop(\PYZsq{}[none]\PYZsq{}, axis=0).sort\PYZus{}values()}
        \PY{n}{tags\PYZus{}frequency} \PY{o}{=} \PY{n}{tags\PYZus{}frequency}\PY{o}{.}\PY{n}{sort\PYZus{}values}\PY{p}{(}\PY{p}{)}

        \PY{n}{axes}\PY{p}{[}\PY{n+nb}{int}\PY{p}{(}\PY{n}{index}\PY{o}{/}\PY{l+m+mi}{2}\PY{p}{)}\PY{p}{,} \PY{n}{index}\PY{o}{\PYZpc{}}\PY{k}{2}].set\PYZus{}title(\PYZdq{}\PYZob{}\PYZcb{}\PYZdq{}.format(month))
        \PY{n}{tags\PYZus{}frequency}\PY{o}{.}\PY{n}{plot}\PY{p}{(}\PY{n}{kind}\PY{o}{=}\PY{l+s+s1}{\PYZsq{}}\PY{l+s+s1}{barh}\PY{l+s+s1}{\PYZsq{}}\PY{p}{,} 
                            \PY{n}{figsize}\PY{o}{=}\PY{p}{(}\PY{l+m+mi}{20}\PY{p}{,}\PY{l+m+mi}{20}\PY{p}{)}\PY{p}{,}
                            \PY{n}{ax}\PY{o}{=}\PY{n}{axes}\PY{p}{[}\PY{n+nb}{int}\PY{p}{(}\PY{n}{index}\PY{o}{/}\PY{l+m+mi}{2}\PY{p}{)}\PY{p}{,} \PY{n}{index}\PY{o}{\PYZpc{}}\PY{k}{2}])

    \PY{n}{figure}\PY{o}{.}\PY{n}{suptitle}\PY{p}{(}\PY{l+s+s1}{\PYZsq{}}\PY{l+s+s1}{Tags les plus populaire par mois en France}\PY{l+s+s1}{\PYZsq{}}\PY{p}{)}
    \PY{n}{plt}\PY{o}{.}\PY{n}{xlabel}\PY{p}{(}\PY{l+s+s2}{\PYZdq{}}\PY{l+s+s2}{nombre de tags}\PY{l+s+s2}{\PYZdq{}}\PY{p}{)}
    \PY{n}{plt}\PY{o}{.}\PY{n}{show}\PY{p}{(}\PY{p}{)}
\end{Verbatim}
\end{tcolorbox}

    \begin{tcolorbox}[breakable, size=fbox, boxrule=1pt, pad at break*=1mm,colback=cellbackground, colframe=cellborder]
\prompt{In}{incolor}{56}{\boxspacing}
\begin{Verbatim}[commandchars=\\\{\}]
\PY{n}{display\PYZus{}tag\PYZus{}per\PYZus{}country}\PY{p}{(}\PY{n}{videos\PYZus{}cleanfr}\PY{p}{)}
\end{Verbatim}
\end{tcolorbox}

    \begin{center}
    \adjustimage{max size={0.9\linewidth}{0.9\paperheight}}{projet_files/projet_39_0.png}
    \end{center}
    { \hspace*{\fill} \\}
    
    \begin{tcolorbox}[breakable, size=fbox, boxrule=1pt, pad at break*=1mm,colback=cellbackground, colframe=cellborder]
\prompt{In}{incolor}{57}{\boxspacing}
\begin{Verbatim}[commandchars=\\\{\}]
\PY{n}{display\PYZus{}tag\PYZus{}per\PYZus{}country}\PY{p}{(}\PY{n}{videos\PYZus{}cleande}\PY{p}{)}
\end{Verbatim}
\end{tcolorbox}

    \begin{center}
    \adjustimage{max size={0.9\linewidth}{0.9\paperheight}}{projet_files/projet_40_0.png}
    \end{center}
    { \hspace*{\fill} \\}
    
    \begin{tcolorbox}[breakable, size=fbox, boxrule=1pt, pad at break*=1mm,colback=cellbackground, colframe=cellborder]
\prompt{In}{incolor}{58}{\boxspacing}
\begin{Verbatim}[commandchars=\\\{\}]
\PY{n}{display\PYZus{}tag\PYZus{}per\PYZus{}country}\PY{p}{(}\PY{n}{videos\PYZus{}cleangb}\PY{p}{)}
\end{Verbatim}
\end{tcolorbox}

    \begin{center}
    \adjustimage{max size={0.9\linewidth}{0.9\paperheight}}{projet_files/projet_41_0.png}
    \end{center}
    { \hspace*{\fill} \\}
    
    \hypertarget{conclusion-sur-les-tags}{%
\subsubsection{3.3.4 Conclusion sur les
Tags}\label{conclusion-sur-les-tags}}

Le résultat concernant le tag {[}none{]} reste le même pour la France et
l'Allemagne néanmoins on peut remarquer que en Grande Bretagne les tags
les plus populaires sont souvent en liens avec l'humour (``funny'',
``comedy''), les autres tags sont cette fois bien différent celon le
mois que l'on visionne, il semble donc bien y a voir un liens entre la
periode de publication et les tags utlisés, on voit par exemple que lors
de la période où Johnny Haliday est mort, en France, le tag ``Johnny
Halliday'' est le 3eme plus utilisé. Pourtant ce résultat est bien
spécifique à la France on pourra ensuite comparer ces résultat avec les
résultat des dataframes d'autres pays pour voir si certains tags sont
internationals.

    \hypertarget{analyse-des-catuxe9gories}{%
\subsubsection{3.3.5 Analyse des
catégories}\label{analyse-des-catuxe9gories}}

Dans la cellule suivante le résultat obtenue est un classement des
catégorie les plus vues, encore une fois en France. On pourra plus tard
lié les tags au catégorie pour permettre de savoir quels sont les tags
les plus populaires et efficace dans chaque catégorie.

    \begin{tcolorbox}[breakable, size=fbox, boxrule=1pt, pad at break*=1mm,colback=cellbackground, colframe=cellborder]
\prompt{In}{incolor}{59}{\boxspacing}
\begin{Verbatim}[commandchars=\\\{\}]
\PY{n}{cat\PYZus{}per\PYZus{}likes} \PY{o}{=} \PY{n}{videos\PYZus{}clean}\PY{o}{.}\PY{n}{filter}\PY{p}{(}\PY{p}{[}\PY{l+s+s1}{\PYZsq{}}\PY{l+s+s1}{category\PYZus{}id}\PY{l+s+s1}{\PYZsq{}}\PY{p}{,} \PY{l+s+s1}{\PYZsq{}}\PY{l+s+s1}{views}\PY{l+s+s1}{\PYZsq{}}\PY{p}{]}\PY{p}{)}
\PY{n}{category\PYZus{}grouped} \PY{o}{=} \PY{n}{cat\PYZus{}per\PYZus{}likes}\PY{o}{.}\PY{n}{groupby}\PY{p}{(}\PY{n}{by}\PY{o}{=}\PY{l+s+s1}{\PYZsq{}}\PY{l+s+s1}{category\PYZus{}id}\PY{l+s+s1}{\PYZsq{}}\PY{p}{)}\PY{o}{.}\PY{n}{sum}\PY{p}{(}\PY{p}{)}\PY{o}{.}\PY{n}{sort\PYZus{}values}\PY{p}{(}\PY{n}{by}\PY{o}{=}\PY{l+s+s1}{\PYZsq{}}\PY{l+s+s1}{views}\PY{l+s+s1}{\PYZsq{}}\PY{p}{,} \PY{n}{ascending}\PY{o}{=}\PY{k+kc}{True}\PY{p}{)}

\PY{n}{category\PYZus{}grouped} \PY{o}{=} \PY{n}{category\PYZus{}grouped}\PY{o}{.}\PY{n}{reset\PYZus{}index}\PY{p}{(}\PY{p}{)}

\PY{n}{category\PYZus{}grouped}\PY{p}{[}\PY{l+s+s1}{\PYZsq{}}\PY{l+s+s1}{category\PYZus{}id}\PY{l+s+s1}{\PYZsq{}}\PY{p}{]} \PY{o}{=} \PY{n}{category\PYZus{}grouped}\PY{p}{[}\PY{l+s+s1}{\PYZsq{}}\PY{l+s+s1}{category\PYZus{}id}\PY{l+s+s1}{\PYZsq{}}\PY{p}{]}\PY{o}{.}\PY{n}{astype}\PY{p}{(}\PY{n+nb}{str}\PY{p}{)}

\PY{k}{with} \PY{n+nb}{open}\PY{p}{(}\PY{l+s+s1}{\PYZsq{}}\PY{l+s+s1}{FR\PYZus{}category\PYZus{}id.json}\PY{l+s+s1}{\PYZsq{}}\PY{p}{,} \PY{l+s+s1}{\PYZsq{}}\PY{l+s+s1}{r}\PY{l+s+s1}{\PYZsq{}}\PY{p}{)} \PY{k}{as} \PY{n}{category\PYZus{}file}\PY{p}{:}
    \PY{n}{category\PYZus{}data} \PY{o}{=} \PY{p}{(}\PY{n}{json}\PY{o}{.}\PY{n}{load}\PY{p}{(}\PY{n}{category\PYZus{}file}\PY{p}{)}\PY{p}{)}\PY{p}{[}\PY{l+s+s1}{\PYZsq{}}\PY{l+s+s1}{items}\PY{l+s+s1}{\PYZsq{}}\PY{p}{]}
    
\PY{n}{category\PYZus{}id} \PY{o}{=} \PY{p}{\PYZob{}}\PY{p}{\PYZcb{}}
\PY{k}{for} \PY{n}{cat} \PY{o+ow}{in} \PY{n}{category\PYZus{}data}\PY{p}{:}
    \PY{n}{category\PYZus{}id}\PY{p}{[}\PY{n}{cat}\PY{p}{[}\PY{l+s+s1}{\PYZsq{}}\PY{l+s+s1}{id}\PY{l+s+s1}{\PYZsq{}}\PY{p}{]}\PY{p}{]} \PY{o}{=}  \PY{n}{cat}\PY{p}{[}\PY{l+s+s1}{\PYZsq{}}\PY{l+s+s1}{snippet}\PY{l+s+s1}{\PYZsq{}}\PY{p}{]}\PY{p}{[}\PY{l+s+s1}{\PYZsq{}}\PY{l+s+s1}{title}\PY{l+s+s1}{\PYZsq{}}\PY{p}{]}

\PY{n}{category\PYZus{}grouped}\PY{p}{[}\PY{l+s+s1}{\PYZsq{}}\PY{l+s+s1}{category\PYZus{}id}\PY{l+s+s1}{\PYZsq{}}\PY{p}{]} \PY{o}{=} \PY{n}{category\PYZus{}grouped}\PY{p}{[}\PY{l+s+s1}{\PYZsq{}}\PY{l+s+s1}{category\PYZus{}id}\PY{l+s+s1}{\PYZsq{}}\PY{p}{]}\PY{o}{.}\PY{n}{map}\PY{p}{(}\PY{n}{category\PYZus{}id}\PY{p}{)}

\PY{n}{category\PYZus{}grouped}\PY{o}{.}\PY{n}{plot}\PY{p}{(}\PY{n}{kind}\PY{o}{=}\PY{l+s+s1}{\PYZsq{}}\PY{l+s+s1}{barh}\PY{l+s+s1}{\PYZsq{}}\PY{p}{,}
                      \PY{n}{y}\PY{o}{=}\PY{l+s+s1}{\PYZsq{}}\PY{l+s+s1}{views}\PY{l+s+s1}{\PYZsq{}}\PY{p}{,}
                      \PY{n}{x}\PY{o}{=}\PY{l+s+s1}{\PYZsq{}}\PY{l+s+s1}{category\PYZus{}id}\PY{l+s+s1}{\PYZsq{}}\PY{p}{,}
                      \PY{n}{figsize}\PY{o}{=}\PY{p}{(}\PY{l+m+mi}{10}\PY{p}{,}\PY{l+m+mi}{10}\PY{p}{)}\PY{p}{,}
                      \PY{n}{title}\PY{o}{=}\PY{l+s+s1}{\PYZsq{}}\PY{l+s+s1}{Most viewed category in trending videos}\PY{l+s+s1}{\PYZsq{}}\PY{p}{,}
                     \PY{p}{)}
\PY{n}{plt}\PY{o}{.}\PY{n}{xlabel}\PY{p}{(}\PY{l+s+s2}{\PYZdq{}}\PY{l+s+s2}{nombre de tags}\PY{l+s+s2}{\PYZdq{}}\PY{p}{)}
\PY{n}{plt}\PY{o}{.}\PY{n}{show}\PY{p}{(}\PY{p}{)}
\end{Verbatim}
\end{tcolorbox}

    \begin{center}
    \adjustimage{max size={0.9\linewidth}{0.9\paperheight}}{projet_files/projet_44_0.png}
    \end{center}
    { \hspace*{\fill} \\}
    
    Dans la cellule suivante, on essaye de visualiser les mots les plus
utilisés dans les titres des videos en tendance, cela n'est qu'une
ébauche.


    % Add a bibliography block to the postdoc
    
    
    
\end{document}
